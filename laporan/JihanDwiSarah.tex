\newthought{\textbf{Jihan Dwi Sarah - 2020903430015 - TRKJ 3B}}


\newday{\textbf{1 - 2 Desember 2022} - Instalasi Hadoop}
\begin{enumerate}
\item Kendala dan Solusi
% jelaskan kendala dan penyebab yang dialami saat mengikuti praktikum serta solusi atau langkah-langkah yang telah dilakukan
\newline Pada pertemuan hari ini, kegiatan yang dilakukan adalah menginstall Apache Hadoop. Selama praktikum tidak mengalami kendala.

\item Kesimpulan \\
% berikan kesimpulan dari praktikum yang telah dikerjkan
Berhasil melakukan instalasi java tanpa ada bug atau error serta instalasi hadoop berikut ini gambar hasil verifikasi instalasi java version dan hadoop version 

\begin{figure}[!ht]
\includegraphics[width=\textwidth]{JihanDwiSarah/Java-version(Jihan)}
\caption{Verifikasi Hasil Instalasi Java}
\label{gam:Java-version(Jihan)}
\end{figure} 

\begin{figure}[!ht]
\includegraphics[width=\textwidth]{JihanDwiSarah/Hadoop-version(Jihan)}
\caption{Verifikasi Hasil Instalasi Hadoop}
\label{gam:Hadoop-version(Jihan)}
\end{figure}
\end{enumerate}


\newday{\textbf{8 - 9 Desember 2022} - Konfigurasi Hadoop}
\begin{enumerate}
\item Kendala dan Solusi \\
% jelaskan kendala dan penyebab yang dialami saat mengikuti praktikum serta solusi atau langkah-langkah yang telah dilakukan
Pada pertemuan hari ini, kegiatan yang dilakukan adalah mengkonfigurasi Apache Hadoop. Selama praktikum tidak mengalami kendala.

\item Kesimpulan \\
% berikan kesimpulan dari praktikum yang telah dikerjkan
Berhasil mengkonfigurasi beberapa file Hadoop sehingga memudahkan dalam memonitoring ekosistem Hadoop yang telah diinstall. Berikut ini gambar bukti keberhasilan praktikum. 

\begin{figure}[!ht]
\includegraphics[width=\textwidth]{JihanDwiSarah/perintah-jps(jihan)}
\caption{Hasil perintah jps}
\label{gam:perintah-jps(jihan)}
\end{figure} 


\begin{figure}[!ht]
\includegraphics[width=\textwidth]{JihanDwiSarah/Akses-web-browser-8088(Jihan)}
\caption{Akses melalui web browser dengan alamat http://localhost:8088}
\label{gam:Akses-web-browser-8088(Jihan)}
\end{figure} 

\begin{figure}[!ht]
\includegraphics[width=\textwidth]{JihanDwiSarah/Akses-web-browser-9870(Jihan)}
\caption{Akses melalui web browser dengan alamat http://localhost:9870}
\label{gam:Akses-web-browser-9870(Jihan)}
\end{figure} 
\end{enumerate}

\newday{\textbf{15 Desember 2022} - WordCount Bawaan Hadoop}
\begin{enumerate}
\item Kendala dan Solusi \\
% jelaskan kendala dan penyebab yang dialami saat mengikuti praktikum serta solusi atau langkah-langkah yang telah dilakukan
Pada pertemuan hari ini, kegiatan yang dilakukan adalah mencoba program bawaan Hadoop untuk memahami bagaimana
proses dan cara kerja Hadoop dalam memproses data input hingga menghasilkan sebuah output. Selama praktikum tidak mengalami kendala.

\item Kesimpulan\\
% berikan kesimpulan dari praktikum yang telah dikerjkan
Berhasil mencoba program bawaan Hadoop yaitu program menghitung jumlah kata dalam data input yang diberikan.Berikut ini gambar bukti keberhasilan praktikum. 
\begin{figure}[!ht]
\includegraphics[width=\textwidth]{JihanDwiSarah/WordCount bawaan-Hadoop(jihan)}
\caption{Output Wordcount Bawaan Hadoop}
\label{gam:WordCount bawaan-Hadoop(jihan)}
\end{figure}
\end{enumerate}

\newday{\textbf{16 Desember 2022} - Program WordCount dengan Java}
\begin{enumerate}
\item Kendala dan Solusi \\
% jelaskan kendala dan penyebab yang dialami saat mengikuti praktikum serta solusi atau langkah-langkah yang telah dilakukan
Pada pertemuan hari ini, kegiatan yang dilakukan adalah mencoba program  wordcount dengan java. Selama praktikum mengalami kendala pada poin ke-6 berdasarkan urutan di modul. Program tidak mau dicompile karena kesalahan penulisan perintah.\\

solusinya adalah menggunakan perintah seperti berikut \\
\begin{figure}[!ht]
\includegraphics[width=\textwidth]{JihanDwiSarah/solusi-java-compile(jihan)}
\caption{Solusi Meng-Compile java}
\label{gam:solusi-java-compile(jihan)}
\end{figure}


\item Kesimpulan\\
% berikan kesimpulan dari praktikum yang telah dikerjkan
Dapat memberikan pemahaman mengenai proses membuat program wordcount java, menyiapkan data, meng-compile program hingga menjalankan program dan memperoleh hasilnya. Berikut hasil praktikum.

\begin{figure}[!ht]
\includegraphics[width=\textwidth]{JihanDwiSarah/WordCount-Java(jihan)}
\caption{Output Wordcount java}
\label{gam:WordCount-Java(jihan)}
\end{figure}
\end{enumerate}

\newday{\textbf{17 Desember 2022} - Instalasi Apache Spark (PySpark)}
\begin{enumerate}
\item Kendala dan Solusi \\
% jelaskan kendala dan penyebab yang dialami saat mengikuti praktikum serta solusi atau langkah-langkah yang telah dilakukan
Terdapat kendala pada poin ke 3 karena kesalahan dari modulnya, perintah yang benar adalah sebagai berikut:\\
sudo mv spark-3.3.1-bin-hadoop3.\\
tidak perlu menambahkan '/usr/local/spark' lagi di akhir.


\item Kesimpulan\\
% berikan kesimpulan dari praktikum yang telah dikerjkan
Berhasil melakukan instalasi apache spark (PySpark), berikut hasil praktikum : \\

\begin{figure}[!ht]
\includegraphics[width=\textwidth]{JihanDwiSarah/Instalasi-Spark(jihan)}
\caption{Hasil Instalasi Spark}
\label{gam:Instalasi-Spark(jihan)}
\end{figure}


\end{enumerate}


\newday{\textbf{22 Desember 2022} - Program WordCount dengan Python }
\begin{enumerate}
\item Kendala dan Solusi \\
% jelaskan kendala dan penyebab yang dialami saat mengikuti praktikum serta solusi atau langkah-langkah yang telah dilakukan
Terdapat error di poin ke-6 yaitu mencoba program local. Belum ada solusinya 

\item Kesimpulan\\
% berikan kesimpulan dari praktikum yang telah dikerjkan
Belum ada hasil akhir, karena masih ada kendala. Masih mencoba sampai berhasil.
\begin{figure}[!ht]
\includegraphics[width=\textwidth]{JihanDwiSarah/kendala-WordCountPython(jihan)}
\caption{Kendala Program Local}
\label{gam:kendala-WordCountPython(jihan)}
\end{figure}


\end{enumerate}

\newday{\textbf{23 Desember 2022} - Program WordCount dengan Pyspark }
\begin{enumerate}
\item Kendala dan Solusi \\
% jelaskan kendala dan penyebab yang dialami saat mengikuti praktikum serta solusi atau langkah-langkah yang telah dilakukan

\item Kesimpulan\\
% berikan kesimpulan dari praktikum yang telah dikerjkan

\end{enumerate}

