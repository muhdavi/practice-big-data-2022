%%% Laboratory	 Notes
%%% Template by Mikhail Klassen, April 2013
%%% Contributions from Sarah Mount, May 2014
%%% Contributions from Muhammad Davi, September 2022
\documentclass[a4paper]{tufte-handout}
\usepackage{lab_notes}

\title{Practice Big Data}
\date{2022}

\begin{document}
\maketitle

%%%%%%%%%%%%%%%%%%%%%%%%%%%%%%%%%%%%%%%%%%%%%%%%%%%%%%%%

\begin{projects}
	\begin{description}
		\item [Muhammad Davi, S.Kom., M.Cs.] adalah sebagai dosen pengampu matakuliah practice big data\footnote{Dosen Prodi Teknologi Rekayasa Komputer Jaringan, Jurusan Teknologi Informasi dan Komputer, Politeknik Negeri Lhokseumawe}.
		\item [Peserta dan Kelompok] matakuliah practice big data adalah sebegai berikut:

\begin{table}[!ht]
\caption{Peserta dan Kelompok Matakuliah Practice Big Bata}
\label{tab:peserta}
\centering
\begin{tabular}{ll} 
\toprule
Nama &	Akun Github\\
\midrule
Kelompok 1\\
\midrule
Nadzura Kumaira			& \url{https://github.com/NadzuraKumaira} \\
Nurani Harum Fardaniah	& \url{https://github.com/fardaniahnh} \\
Nuraula Tafiza			& \url{https://github.com/olala17} \\
Nurul Aflah				& \url{https://github.com/Nurulaflahhh} \\
Faiza Yuwafiqi			& \url{https://github.com/faizayuwafiqi} \\
\midrule
Kelompok 2\\
\midrule
Adinda Awaliah			& \url{https://github.com/AdindaAwaliah} \\
Adjie Yusmunandar		& \url{https://github.com/AdjieYusmunandar} \\
Arya Saputra			& \url{https://github.com/AryaSpt} \\
Jihan Dwi Sarah			& \url{https://github.com/jhndsrh} \\
\midrule
Kelompok 3\\
\midrule
Muhammad Munawir		& \url{https://github.com/Munawir027} \\
Muhammad Ikrammullah	& \url{https://github.com/Ikram160302} \\
M. Ikhsan				& \url{https://github.com/Muhammadikhsandev} \\
Zulfahmi				& \url{https://github.com/zulfahmidev} \\
\midrule
Kelompok 4\\
\midrule
Salsabila Irmanda		& \url{https://github.com/salsabilairmanda17} \\
\textcolor{red}{Siti Hajar Al Zahra	}	& - \\
Syarfani Akbar			& \url{https://github.com/SyarfaniAkbar} \\
Cut Opy Mandalisa		& \url{https://github.com/cutopymdl} \\
\midrule
Kelompok 5\\
\midrule
Rauzatinur Syah			& \url{https://github.com/rauzatinursyah} \\
Resha Russita			& \url{https://github.com/resharussita} \\
Rizki Ilhami			& \url{https://github.com/RIZKIINC} \\
Taravia Fauzah			& \url{https://github.com/traviafzah} \\
\midrule
\end{tabular}
\end{table}
	\end{description}
\end{projects}

%%%%%%%%%%%%%%%%%%%%%%%%%%%%%%%%%%%%%%%%%%%%%%%%%%%%%%%%

\begin{maybe}
\begin{itemize}
\item Waktu, Tempat dan Penilaian

\begin{itemize}
\item Waktu: 13.30 - 16.00, 16.20 - 18.00\footnote{Istirahat dan Sholat Ashar 20 Menit}
\item Tempat: Ruang Lab 3\footnote{Lab Jaringan dan Multimedia di Lantai Dasar Gedung Utama}
\item Penilaian\footnote{Sesuai ketentuan dari Kepala Lab}

\begin{multicols}{2}
\begin{itemize}
\item Responsi Kompetensi
\item Sikap
\item Laporan
\item Seminar
\item UAS
\item Hasil/Benda kerja
\end{itemize}
\end{multicols}
\end{itemize}

\item Sebelum masuk lab wajib berbaris dan berdo'a terlebih dahulu di depan lab.
\item Sebelum perkuliahan dimulai, mahasiswa atau yang mewakili memberi laporan.
\item Setiap keluar lab meminta izin kepada dosen pengampu.
\item Mengikuti Tata Tertib yang berlaku.
\item Referensi

\begin{itemize}
\item Buku Ajar Big Data \citep{Mursyidah2020}.
\end{itemize}

\item Tata Cara Presensi selama Perkuliahan Online

\begin{itemize}
\item Buka repository laporan-practice-big-data melalui akun GitHub masing-masing
\item Lakukan sinkro fork (Sync fork)
\item Buka file {\tt lab\_notes.tex} menggunakan texmaker dan buka file laporan masing-masing
\item Tambahkan kode berikut dan sesuaikan tanggalnya. \\
\textbf{\textcolor{red}{Pastikan kode sudah benar!}}
\begin{lstlisting}
\newday{\textbf{1 Desember 2022}}
\begin{enumerate}
	\item Kendala dan Solusi
	\item Kesimpulan
\end{enumerate}
\end{lstlisting}
\item Lakukan {\tt git add, git commit dan git push}. \\ 
Format \textit{message} saat {\tt git commit} \textit{"Presensi dd-mm-yyyy a.n. Nama"}. Contoh: "Presensi 01-12-2022 a.n. Davi"
\item Buat {\tt pull requests} melalui menu Contribute
\end{itemize}
\end{itemize}
\end{maybe}

%%%%%%%%%%%%%%%%%%%%%%%%%%%%%%%%%%%%%%%%%%%%%%%%%%%%%%%%
\clearpage
\newday{\#1 - 8 September 2022}

\newthought{Introduction \& Preparation}

Pada pertemuan pertama, kegiatan lab adalah perkenalan dan persiapan kebutuhan untuk praktik big data. Setelah dosen pengampu memperkenalkan diri dan matakuliah yang diajarkan, dilanjutkan perkenalan dari setiap mahasiswa dah hasilnya dapat dilihat pada Tabel \ref{tab:perkenalan}.

\begin{table}[!ht]
\vspace*{.5cm}
\caption{Hasil Perkenalan Mahasiswa}
\label{tab:perkenalan}
\centering
\begin{tabular}{cllr} 
\toprule
No & Nama 					& Asal Sekolah 			& Alamat\\
\midrule
1 & Adinda Awaliah			& SMA N 1 Lhokseumawe 	& Cunda \\
2 & Adjie Yusmunandar		& SMK N 1 Lhokseumawe 	& Paloh Lada \\
3 & Arya Saputra			& & \\
4 & Cut Opy Mandalisa		& SMA N 1 Syamtalira Bayu	& Bayu \\
5 & Faiza Yuwafiqi			& & \\
6 & Jihan Dwi Sarah			& SMA N 1 Lhokseumawe 	& Panggoi \\
7 & M. Ikhsan				& SMK N 1 Simpang Kiri 	& Subulussalam \\
\midrule
8 & Muhammad Ikrammullah	& & \\
9 & Muhammad Munawir		& SMK N 1 Lhoksukon		& Karing Meurah Mulia \\
10 & Nadzura Kumaira		& SMK N 2 Lhokseumawe 	& Keude Aceh \\
11 & Nurani Harum Fardaniah	& SMK N 1 Lhoksukon 	& Buket Hagu \\
12 & Nuraula Tafiza			& SMK N 1 Lhoksukon 	& Alue Buket \\
13 & Nurul Aflah			& & \\
14 & Rauzatinur Syah		& MAS Misbahul Ulum 	& Geudong \\
\midrule
15 & Resha Russita			& SMA N 1 Lhokseumawe 	& Alue Awe \\
16 & Rizki Ilhami			& SMK N 1 Lhoksukon 	& Lapang \\
17 & Salsabila Irmanda		& MAS Misbahul Ulum 	& Alue Awee \\
18 & Siti Hajar Al Zahra	& & \\
19 & Syarfani Akbar			& & \\
20 & Taravia Fauzah			& SMA N 1 Dewantara 	& Blang Naleung Mameh \\
21 & Zulfahmi				& SMK N 1 Lhokseumawe	& Kuta Makmur \\
\bottomrule
\end{tabular}
\end{table}

\vspace*{.5cm}
Setelah perkenalan, setiap mahasiswa membuat akun github dan akun discord sebagai media komunikasi dan tempat bekerja secara berkelompok. Hasil dari kegiatan tersebut dapat dilihat pada Tabel \ref{tab:peserta} dan Server Discord pada alamat \url{https://discord.gg/V32EWUx6}.

Tugas di pertemuan pertama adalah menyiapkan \textit{environment} untuk tempat kerja minimal sebagai berikut:
\begin{multicols}{2}
\begin{itemize}
\setlength\itemsep{0em}
\item Processor 2 GHz dual-core
\item RAM sebesar 4 GB
\item Harddisk kosong 25 GB
\item Resolusi layar 1024 x 768
\item \textit{Operating System}: Ubuntu 22.04 LTS
\end{itemize}
\end{multicols}
\hrulefill

%%%%%%%%%%%%%%%%%%%%%%%%%%%%%%%%%%%%%%%%%%%%%%%%%%%%%%%%
\clearpage
\newday{\#2 - 15 September 2022}

\newthought{Instalasi Apache Hadoop}

Pada pertemuan kedua ini kegiatan yang dilakukan adalah menginstall Apache Hadoop pada \textit{environment} yang telah dibuat pada pertemuan sebelumnya. Untuk menginstall Apache Hadoop dapat mengikut langkah-langkah berikut ini:

\begin{enumerate}
\item Membuat Group dan User Baru
\begin{itemize}
\item Membuat group \\
{\tt sudo addgroup hadoop}
\item Membuat User Baru dan Menambahkan ke Group \\
{\tt sudo adduser -ingroup hadoop hdfs}
\item Ubah Hak Akses \\
{\tt sudo visudo}
\item Tambahkan Kode \\
{\tt hdfs	ALL=(ALL:ALL) ALL}
\item Ganti ke User Baru \\
{\tt su - hdfs}
\end{itemize}

\item Install Java \\
{\tt sudo apt update} \\
{\tt sudo apt install openjdk-8-jdk -y} \\

\item Verifikasi Hasil Instalasi Java \\
{\tt java -version} \\
Jika instalasi java berhasil tanpa ada bug atau error maka akan menampilkan hasil seperti pada Gambar \ref{gam:java-version}.

\begin{figure}
\setlength{\belowcaptionskip}{-10pt}
\includegraphics[width=\textwidth]{java-version}
\caption{Versi Java yang Terinstall}
\label{gam:java-version}
\end{figure}

\item Setting SSH \\
\begin{itemize}
\item Uninstall OpenSSH \\
{\tt sudo apt remove openssh-server openssh-client}
\item Install OpenSSH Baru \\
{\tt sudo apt update} \\
{\tt sudo apt install openssh-server openssh-client} \\
{\tt sudo ufw allow 22} \\
{\tt sudo systemctl restart ssh} \\
{\tt sudo apt install ssh} \\
{\tt sudo apt install rsync}
\item Generate Key \\
{\tt ssh-keygen -t dsa -P '' -f $\sim$/.ssh/id\_dsa} \\
{\tt cat $\sim$/.ssh/id\_dsa.pub >> $\sim$/.ssh/authorized\_keys} \\
{\tt ssh-keygen -t rsa}
\item Coba Masuk via SSH \\
{\tt ssh localhost}
\item Jika masih harus memasukkan password, lanjutkan langakh berikut. \\
{\tt ssh-keygen -t rsa} \\
{\tt cat $\sim$/.ssh/id\_rsa.pub >> $\sim$/.ssh/authorized\_keys} \\
{\tt chmod og-wx $\sim$/.ssh/authorized\_keys} \\
{\tt sudo apt-get update}
\item Coba Masuk via SSH Lagi \\
{\tt ssh localhost}
\item Jika sudah berhasil, keluar dengan perintah {\tt exit} \\
{\tt exit}
\end{itemize}

\item Download Apache Hadoop \\
{\tt wget https://dlcdn.apache.org/hadoop/common/hadoop-3.3.4/hadoop-3.3.4.tar.gz}

\item Ekstrak Apache Hadoop \\
{\tt tar -xzvf hadoop-3.3.4.tar.gz }
\begin{itemize}
\item x $\Rightarrow$ ekstrak file arsip.
\item z $\Rightarrow$ filter file arsip melalui gzip.
\item v $\Rightarrow$ menampilkan proses.
\item f $\Rightarrow$ nama file arsip.
\end{itemize}

\item Pindahkan hasil ekstraksi ke {\tt /usr/local/} \\
{\tt sudo mv hadoop-3.3.4 /usr/local/hadoop}

\item Ubah Hak Akses {\tt /usr/local/hadoop} \\
{\tt sudo chown -R hdfs:hadoop /usr/local/hadoop} \\
{\tt sudo chmod -R 777 /usr/local/hadoop}

\item Edit file {\tt sysctl.conf} untuk Disable IPV6
\begin{itemize}
\item Buka file {\tt sysctl.conf} \\
{\tt sudo nano /etc/sysctl.conf}
\item Tambahkan Kode
\begin{lstlisting}
net.ipv6.conf.all.disable_ipv6=1
net.ipv6.conf.lo.disable_ipv6=1
net.ipv6.conf.default_ipv6=1
\end{lstlisting}
\end{itemize}

\item Edit file {\tt hadoop-env.sh} \\
{\tt cd /usr/local/hadoop/etc/hadoop} \\
{\tt sudo nano hadoop-env.sh} \\

\begin{lstlisting}
export JAVA_HOME=/usr/lib/jvm/java-8-openjdk-amd64
export HADOOP_OPTS=-Djava.net.preferIPv4Stack=true
export HADOOP_HOME_WARN_SUPPRESS="TRUE"
export HADOOP_ROOT_LOGGER="WARN"
\end{lstlisting}

\item Menambahkan Hadoop ke {\tt .bashrc}\footnote{Lebih jelasnya lihat Gambar \ref{gam:hadoop-path}} \\
{\tt sudo nano $\sim$/.bashrc} \\
\begin{lstlisting}
#Hadoop Location
export HADOOP_HOME=/usr/local/hadoop
export HADOOP_CONF_DIR=/usr/local/hadoop/etc/hadoop
export HADOOP_MAPRED_HOME=/usr/local/hadoop
export HADOOP_COMMON_HOME=/usr/local/hadoop
export HADOOP_HDFS_HOME=/usr/local/hadoop
export YARN_HOME=/usr/local/hadoop
export PATH=$PATH:/usr/local/hadoop/bin
export PATH=$PATH:/usr/local/hadoop/sbin

#Hadoop Native Location
export HADOOP_COMMON_LIB_NATIVE_DIR=$HADOOP_HOME/lib/native
export HADOOP_OPTS="$HADOOP_OPTS -Djava.library.path=$HADOOP_HOME/lib/native"
export LD_LIBRARY_PATH=$HADOOP_HOME/lib/native
\end{lstlisting}

{\tt source .bashrc}

\item Verifikasi Hasil Instalasi Hadoop \\
{\tt hadoop version} \\
Jika instalasi hadoop berhasil, maka ketika mengecek versi hadoop akan muncul seperti yang diperlihatkan pada Gambar \ref{gam:hadoop-version}.
\begin{figure}[!ht]
\includegraphics[width=\textwidth]{hadoop-version}
\caption{Versi Hadoop yang Terinstall}
\label{gam:hadoop-version}
\end{figure}
\end{enumerate}
 
\hrulefill

%%%%%%%%%%%%%%%%%%%%%%%%%%%%%%%%%%%%%%%%%%%%%%%%%%%%%%%%
\clearpage
\newday{\#3 - 22 September 2022}
\textit{N.B.: Setiap mahasiswa membuat laporan hasil praktik sesuai dengan format yang telah ditentukan. Template laporan dapat di download pada alamat \url{https://github.com/muhdavi/laporan-practice-big-data}.}

\newthought{Konfigurasi Apache Hadoop}

Setelah selesai meng-install Hadoop, kita perlu konfigurasi beberapa file Hadoop agar memudahkan kita dalam memonitoring ekosistem Hadoop yang telah diinstall.

\begin{enumerate}
\item Konfigurasi File Hadoop \\
File-file konfigurasi hadoop berada pada folder {\tt hadoop/etc/hadoop} seperti yang diperlihatkan pada Gambar \ref{gam:file-hadoop}.

{\tt cd /usr/local/hadoop/etc/hadoop} \\

\begin{figure}[!ht]
\includegraphics[width=\textwidth]{file-hadoop}
\caption{File Konfigurasi Hadoop}
\label{gam:file-hadoop}
\end{figure}

Beberapa file yang perlu dikonfigurasi adalah sebebai berikut:
\begin{itemize}

\item core-site.xml
\begin{lstlisting}
<property>
	<name>hadoop.tmp.dir</name>
	<value>/app/hadoop/tmp</value>
</property>
<property>
	<name>fs.default.name</name>
	<value>hdfs://localhost:9000</value>
</property>
\end{lstlisting}

\item hdfs-site.xml
\begin{lstlisting}
<property>
	<name>dfs.replication</name>
	<value>1</value>
</property>
<property>
	<name>dfs.namenode.name.dir</name>
	<value>file:/usr/local/hadoop/yarn_data/hdfs/namenode</value>
</property>
<property>
	<name>dfs.datanode.data.dir</name>
	<value>file:/usr/local/hadoop/yarn_data/hdfs/datanode</value>
</property>
\end{lstlisting}

\item mapred-site.xml
\begin{lstlisting}
<property>
	<name>mapred.framework.name</name>
	<value>yarn</value>
</property>
<property>
	<name>mapreduce.jobhistory.address</name>
	<value>localhost:10020</value>
</property>
\end{lstlisting}

\item yarn-site.xml
\begin{lstlisting}
<property>
	<name>yarn.nodemanager.aux-services</name>
	<value>mapreduce_shuffle</value>
</property>
<property>
	<name>yarn.nodemanager.aux-services.mapreduce.shuffle.class</name>
	<value>org.apache.hadoop.mapred.ShuffleHandler</value>
</property>
\end{lstlisting}
\end{itemize}

\item Membuat Folder Sementara (\textit{Temporary}) untuk HDFS \\
{\tt sudo mkdir -p /app/hadoop/tmp} \\
{\tt sudo chown -R hdfs:hadoop /app/hadoop/tmp} \\
{\tt sudo chmod -R 777 /app/hadoop/tmp}

\item Membuat Folder {\tt namenode} dan {\tt datanode} \\
{\tt sudo mkdir -p /usr/local/hadoop/yarn\_data/hdfs/namenode} \\
{\tt sudo mkdir -p /usr/local/hadoop/yarn\_data/hdfs/datanode} \\
{\tt sudo chown -R hdfs:hadoop /usr/local/hadoop/yarn\_data/hdfs/namenode} \\
{\tt sudo chown -R hdfs:hadoop /usr/local/hadoop/yarn\_data/hdfs/datanode} \\
{\tt sudo chmod -R 777 /usr/local/hadoop/yarn\_data/hdfs/namenode} \\
{\tt sudo chmod -R 777 /usr/local/hadoop/yarn\_data/hdfs/datanode}

\item Jalankan Perintah Format HDFS \\
{\tt hdfs namenode -format}

\item Jalankan Hadoop Service \\
{\tt sbin/start-dfs.sh} \\
{\tt sbin/start-yarn.sh} \\

\item Cek Service yang \textit{Running}
\begin{itemize}
\item Jalankan perintah {\tt jps}
\item Akses melalui web browser dengan alamat \url{http://localhost:9870}\footnote{Seperti yang diperlihatkan pada Gambar \ref{gam:namenode}} atau \url{http://localhost:8088}\footnote{Seperti yang diperlihatkan pada Gambar \ref{gam:resourcemanager}}.
\end{itemize}

\begin{figure}[!ht]
\includegraphics[width=\textwidth]{namenode}
\label{gam:namenode}
\end{figure}

\begin{figure}[!ht]
\includegraphics[width=\textwidth]{resourcemanager}
\caption{Resource Manager Hadoop}
\label{gam:resourcemanager}
\end{figure}
\end{enumerate}

Proses analisis big data sebaiknya menggunakan komputer dengan spesifikasi yang tinggi atau \textit{High Performance Computer (HPC)}. Namun jika tidak memiliki HPC cara lain adalah dengan menggabungkan beberapa komputer untuk melakukan proses analisis yang dikenal dengan \textit{Parallel Computing}.
\hrulefill

%%%%%%%%%%%%%%%%%%%%%%%%%%%%%%%%%%%%%%%%%%%%%%%%%%%%%%%%
\clearpage
\newday{\#4 - 24 November 2022 menggantikan 29 September 2022}

\newthought{Instalasi dan Konfigurasi GIT dan GitHub}

Pada pertemuan kali ini kita belajar install GIT dan konfigurasi GIT dengan GitHub agar dapat menjalankan perintah-perintah GIT melalui lokal dan menyimpan hasil kerjaan kita ke GitHub. Karena pada pertemuan pertama telah membuat akun GitHub, maka pada pertemuan kali ini asumsinya semua sudah memiliki akun GitHub. Setelah itu ikuti beberapa langkah berikut untuk praktikum kali ini.

\begin{enumerate}
\item Install Git \\
Pertema download program Git melalui link ini (\url{https://git-scm.com/download}) sesuai dengan sistem operasi yang digunakan. Bagi pengguna Windows jika proses download sudah selesai, lanjut proses instalasi seperti program windows pada umumnya (Next, Next, Next, sampai selesai).

\item \textit{Generate} SSH-Key \\
Jika instalasi sudah selesai, coba buka Git Bash, maka akan muncul program baru yang mirip dengan Terminal atau Command Prompt (CMD) kita sebut Git Bash. Kemudian jalankan perintah berikut untuk \textit{generate ssh-key}.

{\tt ssh-keygen -t ed25519 -C "email@akun.github"} \\

Ganti {\tt email@akun.github} dengan email yang terdaftar pada akun GitHub. Selanjutnya jika muncul beberapa pertanyaan seperti berikut ini tekan [enter].

{\tt > Enter a file in which to save the key (/Users/YOU/.ssh/id\_ed25519:} \\
{\tt > Enter passphrase (empty for no passphrase):} \\
{\tt > Enter same passphrase again:}

Jika proses diatas berhasil maka terbentuk folder baru dengan nama {\tt .ssh}. Didalam folder tersebut terdapat \textit{private key} dan \textit{public key}. Bukan file \textit{public key} (id\_ed25519.pub) dan \textit{copy} isi dari file tersebut.

\item Menambahkan SSH-Key ke Akun GitHub \\
Untuk menambahkan SSH-Key ke akun GitHub pertama login terlebih dahulu. Setelah berhasil login klik pada gambar profil sehingga tampil menu dropdown seperti pada Gambar \ref{gam:langkah-ssh} nomor 1. Selanjutnya pilih menu \textbf{Setting} sepertin yang ditunukan pada nomor 2. Setelah memilih menu \textbf{Setting} maka muncul halaman setting dengan menu di sidebar sebelah kiri. Pada menu sebelah kiri pilih menu \textbf{SSH and GPG keys} seperti pada nomor 3. Kemudian klik tombol \textbf{New SSH key} maka akan muncul form menambahkan SSH seperti yang diperlihatkan pada Gambar \ref{gam:form-ssh}.

\begin{figure}[!ht]
\includegraphics[width=\textwidth]{gtihub-1}
\caption{Lankah-langkah Menambah SSH-Key di GitHub}
\label{gam:langkah-ssh}
\end{figure}

Kode SSH-Key yang telah di-\textit{copy} pada langkah sebelumnya \textit{paste}-kan kode tersebut pada isian \textbf{Key} dan beri judul SSH-Key pada isian \textbf{Title}. Setelah semua diisi klik tombol \textbf{Add SSH key} untuk menyimpan SSH-Key baru tersebut.

\begin{figure}[!ht]
\includegraphics[width=\textwidth]{github-2}
\caption{Form Penambahan SSH-Key di GitHub}
\label{gam:form-ssh}
\end{figure}

Untuk menguji bahwa penambahan SSH-Key telah berhasil coba lakukan {\tt git push} untuk repositori laporan practice big data dari akun masing-masing. Untuk lebih jelasnya tentang GIT dapat membaca catatan pada link berikut \url{https://muhdavi.github.io/learn-git/}.
\end{enumerate}

\begin{figure}[!ht]
\includegraphics[width=.95\textwidth]{24-11-2022}
\caption{Perkuliahan Daring via GMeet 24-11-2022}
\label{gam:form-ssh}
\end{figure}

%%%%%%%%%%%%%%%%%%%%%%%%%%%%%%%%%%%%%%%%%%%%%%%%%%%%%%%%
\clearpage
\newday{\#5 - 25 November 2022 menggantikan 6 Oktober 2022}

\newthought{Instalasi dan Konfigurasi GIT dan GitHub - Lanjutan}

Mahasiswa yang sudah berhasil konfigurasi Git dengan GitHub dan melakukan {\tt Pull Request}:
\begin{multicols}{2}
\begin{enumerate}
\item Rizki Ilhami
\item Rauzatinur Syah $\star$
\item Taravia Fauzah
\item Resha Russita $\star$
\item Nurani Harum Fardaniah $\star$
\item Adinda Awaliah $\star$
\item Salsabila Irmanda $\star$
\item M. Ikhsan
\item Jihan Dwi Sarah $\star$
\item Cut Opy Mandalisa $\star$
\item Zulfahmi
\item Muhammad Munawir
\item Nuraula Tafiza
\item Muhammad Ikrammullah
\item Nadzura Kumaira
\item Adjie Yusmunandar
\end{enumerate}
\end{multicols}

\noindent
Berikut beberapa mahasiswa yang belum melakukan {\tt Pull Request}:
\begin{multicols}{2}
\begin{enumerate}
\item Arya Saputra
\item Faiza Yuwafiqi
\item Nurul Aflah
\item \textcolor{red}{Siti Hajar Al Zahra}
\item Syarfani Akbar
\end{enumerate}
\end{multicols}

\begin{figure}[!ht]
\includegraphics[width=\textwidth]{25-11-2022}
\caption{Perkuliahan Daring via GMeet 25-11-2022}
\label{gam:form-ssh}
\end{figure}

%%%%%%%%%%%%%%%%%%%%%%%%%%%%%%%%%%%%%%%%%%%%%%%%%%%%%%%%
\clearpage
\newday{\#6 - 1 Desember 2022 menggantikan 13 Oktober 2022\footnote{
Mahasiswa yang hadir:
\begin{enumerate}
\item Adinda Awaliah
\item Arya Saputra $\oplus$
\item Cut Opy Mandalisa
\item Jihan Dwi Sarah
\item M. Ikhsan
\item Muhammad Ikrammullah
\item Muhammad Munawir
\item Nadzura Kumaira
\item Nurani Harum Fardaniah
\item Rauzatinur Syah
\item Resha Russita
\item Rizki Ilhami
\item Salsabila Irmanda
\item Syarfani Akbar
\item Taravia Fauzah
\item Zulfahmi
\end{enumerate}}}


%%%%%%%%%%%%%%%%%%%%%%%%%%%%%%%%%%%%%%%%%%%%%%%%%%%%%%%%
\clearpage
\newday{\#7 - 2 Desember 2022 menggantikan 20 Oktober 2022\footnote{
Mahasiswa yang hadir:
\begin{enumerate}
\item Muhammad Munawir
\item Rizki Ilhami
\item Rauzatinur Syah
\item Salsabila Irmanda
\item Adjie Yusmunandar
\item Nurani Harum Fardaniah
\item Cut Opy Mandalisa
\item Resha Russita
\item Taravia Fauzah
\item Adinda Awaliah
\item Jihan Dwi Sarah
\item M. Ikhsan
\item Zulfahmi
\end{enumerate}}}



%%%%%%%%%%%%%%%%%%%%%%%%%%%%%%%%%%%%%%%%%%%%%%%%%%%%%%%%
\clearpage
\newday{\#8 - 8 Desember 2022 menggantikan 3 November 2022}



%%%%%%%%%%%%%%%%%%%%%%%%%%%%%%%%%%%%%%%%%%%%%%%%%%%%%%%%

\newthought{\textbf{Adinda Awaliah - 2020903430004 - TRKJ 3B}}

\newday{\textbf{1-2 desember 2022} - Instalasi dan Konfigurasi Apache Hadoop}
\begin{enumerate}
\item Kendala dan Solusi
\newline praktikum instalasi apache hadoop tidak ada kendala. konfigurasi apache hadoop terkendala pada saat perintah jps, hasil yang muncul hanya 2jps, solusinya kembali ke langkah ssh, dan untuk menampilkan jps harus berada pada /usr/local/hadoop/etc/hadoop

\item Kesimpulan
\newline berhasil melakukan instalasi. berhasil melakukan konfigurasi.

\begin{figure}[!ht]
\includegraphics[width=\textwidth]{AdindaAwaliah/jps}
\caption{hasil dari jps}
\label{gam:jps}
\end{figure}

\begin{figure}
\includegraphics[width=\textwidth]{AdindaAwaliah/konfigurasi apache hadoop web 9870}
\caption{apache hadoop web 9870}
\label{gam:konfigurasi apache hadoop web 9870}
\end{figure}

\begin{figure}
\includegraphics[width=\textwidth]{AdindaAwaliah/konfigurasi apache hadoop web 8088}
\caption{apache hadoop web 8088}
\label{gam:konfigurasi apache hadoop web 8088}
\end{figure}

\end{enumerate}

\newday{\textbf{ 8-9 desember 2022}}
\begin{enumerate}
\item Kendala dan Solusi
% jelaskan kendala dan penyebab yang dialami saat mengikuti praktikum serta solusi atau langkah-langkah yang telah dilakukan

\item Kesimpulan
% berikan kesimpulan dari praktikum yang telah dikerjkan
\newline lanjutan instalasi apache hadoop

\end{enumerate}


\newday{\textbf{15 desember 2022}- Program WordCount bawaan Hadoop}
\begin{enumerate}
\item Kendala dan Solusi
\newline Pada saat melakukan pratikum tidak ada kendala

\item Kesimpulan
\newline Berhasil melakukan praktikum WordCount bawaan Hadoop

\begin{figure}[!ht]
\includegraphics[width=\textwidth]{AdindaAwaliah/hasil ss no 6 wordcount}
\caption{Cek Hasil WordCount Hadoop}
\label{gam:hasil ss no 6 wordcount}
\end{figure}

\begin{figure}[!ht]
\includegraphics[width=\textwidth]{AdindaAwaliah/hasil ss no 7 wordcount}
\caption{Lihat Hasil WordCount Hadoop}
\label{gam:hasil ss no 7 wordcount}
\end{figure}

\end{enumerate}

\newday{\textbf{16 desember 2022}- Program WordCount dengan Java }
\begin{enumerate}
\item Kendala dan Solusi
\newline Tidak ada kendala saat mengerjakan praktikum

\item Kesimpulan
\newline Berhasil menampiklan hasil WordCount Java

\begin{figure}[!ht]
\includegraphics[width=\textwidth]{AdindaAwaliah/hasil ss no 9 wordcount java}
\caption{Cek hasil WordCount Java}
\label{gam:hasil ss no 9 wordcount java}
\end{figure}

\begin{figure}[!ht]
\includegraphics[width=\textwidth]{AdindaAwaliah/hasil ss no 10 wordcount java}
\caption{Lihat hasil WordCount Java}
\label{gam:hasil ss no 9 wordcount java}
\end{figure}

\end{enumerate}

\newday{\textbf{22 desember 2022}- Instalasi ApacheSpark}
\begin{enumerate}
\item Kendala dan Solusi
\newline Saat mengerjakan praktikum ini tidak ada kendala saat mengerjakan praktikum

\item Kesimpulan
\newline Berhasil melakukan installasi

\begin{figure}[!ht]
\includegraphics[width=\textwidth]{AdindaAwaliah/instalasi apache spark no 5}
\caption{Versi Spark yang terinstall}
\label{gam:instalasi apache spark no 5}
\end{figure}

\end{enumerate}

\newday{\textbf{23 desember 2022}- Program WordCount dengan Python }
\begin{enumerate}
\item Kendala dan Solusi
\newline Terkendala saat perintah menjalankan program dengan hadoop pada ~/WordCountPython/map.py -reducer tidak bisa. Solusinya  di ganti dengan /usr/local/hadoop/etc/hadoop karena sebelumnya saya sudah menjalankan program di dalam /usr/local/hadoop/etc/hadoop

\item Kesimpulan
\newline Berhasil melakukan pratikum wordcount python

\begin{figure}[!ht]
\includegraphics[width=\textwidth]{AdindaAwaliah/hasil ss no 8 wordcount python}
\caption{Cek hasil WordCount Python}
\label{gam:hasil ss no 8 wordcount python}
\end{figure}

\begin{figure}[!ht]
\includegraphics[width=\textwidth]{AdindaAwaliah/hasil ss no 9 wordcount python}
\caption{Lihat hasil WordCount Python}
\label{gam:hasil ss no 9 wordcount python}
\end{figure}

\end{enumerate}


\newday{\textbf{23 desember 2022}- Program WordCount dengan PySpark }
\begin{enumerate}
\item Kendala dan Solusi
\newline Saat bmengerjakan praktikum ini, tidak ditemukan kendala

\item Kesimpulan
\newline Berhasil melakukan pratikum wordcount PySpark

\begin{figure}[!ht]
\includegraphics[width=\textwidth]{AdindaAwaliah/wordcup dengan pyspark no 6}
\caption{Cek hasil WordCount PySpark}
\label{gam:wordcup dengan pyspark no 6}
\end{figure}

\begin{figure}[!ht]
\includegraphics[width=\textwidth]{AdindaAwaliah/wordcup dengan pyspark no 7}
\caption{Lihat hasil WordCount PySprak}
\label{gam:wordcup dengan pyspark no 7}
\end{figure}

\end{enumerate}

\newday{\textbf{05 Januari 2023}- Tugas individu Program Machine Learning dengan Pyspark }
\begin{enumerate}
\item Kendala dan Solusi
\newline a. Kendala pertama pada saat menentukan nilai K 		dengan metode silhoutte, grafik tidak muncul, seharusnya setelah perintah plt.show() akan muncul grafik, belum di temukannya solusi
\newline b. Kendala selanjutnya saat menampilkan hasil Clustering dengan PCA, grafik tidak muncul, seharusnya setelah perintah plt.show() akan muncul grafik dan belum di temukannya solusi

\begin{figure}[!ht]
\includegraphics[width=\textwidth]{AdindaAwaliah/kendala grafik 1 nomor 5}
\caption{kendala pertama grafik tidak muncul}
\label{gam:kendala grafik 1 nomor 5}
\end{figure}


\item Kesimpulan
\newline Pada percobaan praktikum ini berhasil memunculkan tabel, namun tidak berhasil untuk memunculkan grafik. Tabel yang muncul merupakan tabel dari load data iris dan tabel penentuan nilai K asseamble data.

\begin{figure}[!ht]
\includegraphics[width=\textwidth]{AdindaAwaliah/grafik 1}
\caption{Grafik pertama }
\label{gam:grafik 1}
\end{figure}

\begin{figure}[!ht]
\includegraphics[width=\textwidth]{AdindaAwaliah/grafik 2}
\caption{Grafik kedua }
\label{gam:grafik 2}
\end{figure}

\end{enumerate}
\newthought{\textbf{Adjie Yusmunandar - 2020903430005 - TRKJ 3B}}


\newday{\textbf{1 - 2 Desember 2022} - Instalasi dan Konfigurasi Apache Hadoop}
\begin{enumerate}
\item Kendala dan Solusi \\
Kendala :\\
Saat melakukan penginstalan hadoop terdapat masalah pada saat mengecek hadoop version dan  pada saat melakukan konfigurasi hadoop terdapat kendala pada saat melakukan hdfs namenode -format. Perintah ini tidak mau dijalankan karena ada kesalahan dalam penulisan pada file yarn-site.xml.

Solusi :\\
Melakukan perubahan penulisan pada file yarn-site.xml,

\item Kesimpulan
\newline
 penginstalan hadoop dan konfigurasi hadoop berhasil dijalankan sesuai perintah-perintah yang ada.

\begin{figure}
\setlength{\belowcaptionskip}{-10pt}
\includegraphics[width=\textwidth]{Adjie Yusmunandar/instalasi hadoop}
\caption{Versi hadoop yang Terinstall}
\label{gam:hasil instalasi apache hadoop}
\end{figure}

\begin{figure}
\setlength{\belowcaptionskip}{-10pt}
\includegraphics[width=\textwidth]{Adjie Yusmunandar/konfigurasi hadoop}
\caption{Versi hadoop yang Terinstall}
\label{gam:hasil konfigurasi hadoop}
\end{figure}
\end{enumerate}

\newday{\textbf{2 Desember 2022}}
\begin{enumerate}
\item Kendala dan Solusi
% jelaskan kendala dan penyebab yang dialami saat mengikuti praktikum serta solusi atau langkah-langkah yang telah dilakukan

\item Kesimpulan
% berikan kesimpulan dari praktikum yang telah dikerjkan

\end{enumerate}

\newday{\textbf{8 Desember 2022}}
\begin{enumerate}
\item Kendala dan Solusi
% jelaskan kendala dan penyebab yang dialami saat mengikuti praktikum serta solusi atau langkah-langkah yang telah dilakukan

\item Kesimpulan
% berikan kesimpulan dari praktikum yang telah dikerjkan

\end{enumerate}

\newday{\textbf{9 Desember 2022}}
\begin{enumerate}
\item Kendala dan Solusi
% jelaskan kendala dan penyebab yang dialami saat mengikuti praktikum serta solusi atau langkah-langkah yang telah dilakukan

\item Kesimpulan
% berikan kesimpulan dari praktikum yang telah dikerjkan

\end{enumerate}

\newday{\textbf{15 Desember 2022}}
\begin{enumerate}
\item Kendala dan Solusi
% jelaskan kendala dan penyebab yang dialami saat mengikuti praktikum serta solusi atau langkah-langkah yang telah dilakukan

\item Kesimpulan
% berikan kesimpulan dari praktikum yang telah dikerjkan

\end{enumerate}

\newday{\textbf{16 Desember 2022}}
\begin{enumerate}
\item Kendala dan Solusi
% jelaskan kendala dan penyebab yang dialami saat mengikuti praktikum serta solusi atau langkah-langkah yang telah dilakukan

\item Kesimpulan
% berikan kesimpulan dari praktikum yang telah dikerjkan

\end{enumerate}

\newday{\textbf{22 Desember 2022}}
\begin{enumerate}
\item Kendala dan Solusi
% jelaskan kendala dan penyebab yang dialami saat mengikuti praktikum serta solusi atau langkah-langkah yang telah dilakukan

\item Kesimpulan
% berikan kesimpulan dari praktikum yang telah dikerjkan

\end{enumerate}



\newthought{\textbf{Arya Saputra - 2020903430009 - TRKJ 3B}}

\newday{\textbf{01 Desember 2022}}
\begin{enumerate}
\item Kendala dan Solusi
Pada pertemuan hari ini, kegiatan yang dilakukan adalah menginstall Apache Hadoop. Selama praktikum tidak mengalami kendala.


\item Kesimpulan
Berhasil melakukan instalasi hadoop berikut ini gambar hasil verifikasi instalasi hadoop version 

\begin{figure}[!ht]
\includegraphics[width=\textwidth]{AryaSyahputra/hadoop version}
\caption{Verifikasi Hasil Instalasi Hadoop}
\label{gam:Hadoop-version}
\end{figure}
\end{enumerate}

\newday{\textbf{02 Desember 2022}}
\begin{enumerate}
\item Kendala
sudah berhasil menginstall hadoop

\item solusi
menginstal ulang ubuntu

\item Kesimpulan
setelah menginstall ulang ubuntu, instalasi dan konfigurasi hadoop berhasil

\begin{figure}[!ht]
\includegraphics[width=\textwidth]{AryaSyahputra/jps}
\caption{Verifikasi Hasil Instalasi Hadoop}
\label{gam:instalasi-hadoop}
\end{figure}
\begin{figure}[!ht]
\includegraphics[width=\textwidth]{AryaSyahputra/localhos1}
\caption{Verifikasi Hasil Instalasi Hadoop}
\label{gam:instalasi-hadoop}
\end{figure}
\begin{figure}[!ht]
\includegraphics[width=\textwidth]{AryaSyahputra/localhos2}
\caption{Verifikasi Hasil Instalasi Hadoop}
\label{gam:instalasi-hadoop}
\end{figure}
\end{enumerate}

\clearpage
\newday{\textbf{15 Desember 2022}}
\begin{enumerate}
\item Kendala dan Solusi



\item Kesimpulan



\end{enumerate}


\newthought{\textbf{Cut Opy Mandalisa - 2020903430012 - TRKJ 3B}}

\newday{\textbf{1-2 Desember 2022- Instalasi dan Konfigurasi Hadoop}}
\begin{enumerate}
\item Kendala dan Solusi\\
% jelaskan kendala dan penyebab yang dialami saat mengikuti praktikum serta solusi atau langkah-langkah yang telah dilakukan
Pada Praktikum pertama yaitu penginstalan apache hadoop.kendala yang didapat tidak bisa membuka firefox kemudian solusinya dengan menginstal firefox baru.

\begin{figure}[!ht]
\includegraphics[width=\textwidth]{01}
\caption{hasil dari cek hadoop service}
\label{gam:perkuliahan-25-11}
\end{figure}


\begin{figure}[!ht]
\includegraphics[width=\textwidth]{02}
\caption{hasil cek hadoop service}
\label{gam:perkuliahan-25-11}
\end{figure}

\item Kesimpulan\\
% berikan kesimpulan dari praktikum yang telah dikerjkan
Behasil mendownload dan menginstal Apache hadoop dan sudah bisa dijalankan

\end{enumerate}

\newday{\textbf{08 Desember 2022-WordCount bawaan Hadoop}}
\begin{enumerate}

\item Kendala dan Solusi
\newpage
% jelaskan kendala dan penyebab yang dialami saat mengikuti praktikum serta solusi atau langkah-langkah yang telah dilakukan
Pada praktikum ini membuat program WordCount bawaan Hadoop. Pada melakukan praktikum tidak ada kendala hanya erorr dikarenakan salah memasukkan perintah.

\begin{figure}[!ht]
\includegraphics[width=\textwidth]{03}
\caption{hasil WordCount bawaan Hadoop}
\label{gam:perkuliahan-25-11}
\end{figure}

\item Kesimpulan\\
% berikan kesimpulan dari praktikum yang telah dikerjkan
Pada praktikum ini untuk memahami proses cara kerja pada hadoop dalam memproses data input sehingga menghasilkan output.Wordcount merupakan program untuk menghitung jumlah kata dalam input.

\end{enumerate}

\newday{\textbf{09 Desember 2022-WordCount dengan Java}}
\begin{enumerate}
\item Kendala dan Solusi\\
% jelaskan kendala dan penyebab yang dialami saat mengikuti praktikum serta solusi atau langkah-langkah yang telah dilakukan
Pada praktikum ini membuat program WordCount dengan java.Pada saat melakukan praktikum terdapat error akan tetapi erorrnya disebabkan salah memasukkan perintah codingannya.solusinya harus lebih teliti saat memasukkan codingan tersebut.

\begin{figure}[!ht]
\includegraphics[width=\textwidth]{04}
\caption{hasil wordCount dengan java}
\label{gam:perkuliahan-25-11}
\end{figure}

\item Kesimpulan\\
% berikan kesimpulan dari praktikum yang telah dikerjkan
Berhasil menjalankan program WordCount dengan java.

\end{enumerate}

\newday{\textbf{15 Desember 2022}}
\begin{enumerate}
\item Kendala dan Solusi
% jelaskan kendala dan penyebab yang dialami saat mengikuti praktikum serta solusi atau langkah-langkah yang telah dilakukan

\item Kesimpulan
% berikan kesimpulan dari praktikum yang telah dikerjkan

\end{enumerate}

\newday{\textbf{16 Desember 2022}}
\begin{enumerate}
\item Kendala dan Solusi
% jelaskan kendala dan penyebab yang dialami saat mengikuti praktikum serta solusi atau langkah-langkah yang telah dilakukan

\item Kesimpulan
% berikan kesimpulan dari praktikum yang telah dikerjkan

\end{enumerate}
\newthought{\textbf{Faiza Yuwafiqi - 2020903430014 - TRKJ 3B}}

\newday{\textbf{15 Desember 2022}}
\begin{enumerate}
\item Kendala dan Solusi
% jelaskan kendala dan penyebab yang dialami saat mengikuti praktikum serta solusi atau langkah-langkah yang telah dilakukan

\item Kesimpulan
% berikan kesimpulan dari praktikum yang telah dikerjkan

\end{enumerate}


\newthought{\textbf{Jihan Dwi Sarah - 2020903430015 - TRKJ 3B}}

\newday{\textbf{22 September 2022}}
\begin{enumerate}
\item Kendala dan Solusi \\
% jelaskan kendala dan penyebab yang dialami saat mengikuti praktikum serta solusi atau langkah-langkah yang telah dilakukan 
Praktikum pertama yaitu melakukan pull dan push Github. Selama praktikum tidak ada kendala. Yang dilakukan selama praktikum pertama yaitu mengedit file pada Latex yaitu menambahkan asal sekolah dan tempat tinggal.


\item Kesimpulan
% berikan kesimpulan dari praktikum yang telah dikerjkan
\newline Berhasil melakukan pull dan push menggunakan Github. 


\end{enumerate}

\newday{\textbf{1 Desember 2022}}
\begin{enumerate}
\item Kendala dan Solusi
% jelaskan kendala dan penyebab yang dialami saat mengikuti praktikum serta solusi atau langkah-langkah yang telah dilakukan
\newline Praktikum selanjutnya tidak ada kendala. Yang dilakukan selama praktikum adalah mengganti nama beserta NIM di Latex pada halaman laporan praktikum masing-masing.

\item Kesimpulan
% berikan kesimpulan dari praktikum yang telah dikerjkan
\newline Berhasil mengganti nama beserta NIM pada halam laporan.

\end{enumerate}


\newday{\textbf{2 Desember 2022}}
\begin{enumerate}
\item Kendala dan Solusi
% jelaskan kendala dan penyebab yang dialami saat mengikuti praktikum serta solusi atau langkah-langkah yang telah dilakukan
\newline Praktikum selanjutnya tidak ada kendala. Yang dilakukan selama praktikum adalah membuat laporan praktikum serta menambahkan baris baru untuk tiap-tiap tanggal pelaksanaan praktikum.

\item Kesimpulan
% berikan kesimpulan dari praktikum yang telah dikerjkan
\newline Berhasil mengedit halaman laporan praktikum.

\end{enumerate}

\newday{\textbf{8 Desember 2022}}
\begin{enumerate}
\item Kendala dan Solusi
% jelaskan kendala dan penyebab yang dialami saat mengikuti praktikum serta solusi atau langkah-langkah yang telah dilakukan
\newline Pada pertemuan hari ini, kegiatan yang dilakukan adalah menginstall Apache Hadoop. Selama praktikum tidak mengalami kendala.


\item Kesimpulan \\
% berikan kesimpulan dari praktikum yang telah dikerjkan
Berhasil melakukan instalasi java tanpa ada bug atau error serta instalasi hadoop berikut ini gambar hasil verifikasi instalasi java version dan hadoop version 


\begin{figure}[!ht]
\includegraphics[width=\textwidth]{JihanDwiSarah/Java-version(Jihan)}
\caption{Verifikasi Hasil Instalasi Java}
\label{gam:Java-version(Jihan)}
\end{figure} 

\begin{figure}[!ht]
\includegraphics[width=\textwidth]{JihanDwiSarah/Hadoop-version(Jihan)}
\caption{Verifikasi Hasil Instalasi Hadoop}
\label{gam:Hadoop-version(Jihan)}
\end{figure} 



\end{enumerate}

\newday{\textbf{9 Desember 2022}}
\begin{enumerate}
\item Kendala dan Solusi \\
% jelaskan kendala dan penyebab yang dialami saat mengikuti praktikum serta solusi atau langkah-langkah yang telah dilakukan
Pada pertemuan hari ini, kegiatan yang dilakukan adalah mengkonfigurasi Apache Hadoop. Selama praktikum tidak mengalami kendala.

\item Kesimpulan \\
% berikan kesimpulan dari praktikum yang telah dikerjkan
Berhasil mengkonfigurasi beberapa file Hadoop sehingga memudahkan dalam memonitoring ekosistem Hadoop yang telah diinstall. Berikut ini gambar bukti keberhasilan praktikum. 

\begin{figure}[!ht]
\includegraphics[width=\textwidth]{JihanDwiSarah/perintah-jps(jihan)}
\caption{Hasil perintah jps}
\label{gam:perintah-jps(jihan)}
\end{figure} 

\begin{figure}[!ht]
\includegraphics[width=\textwidth]{JihanDwiSarah/Akses-web-browser-8088(Jihan)}
\caption{Akses melalui web browser dengan alamat http://localhost:8088}
\label{gam:Akses-web-browser-8088(Jihan)}
\end{figure} 

\begin{figure}[!ht]
\includegraphics[width=\textwidth]{JihanDwiSarah/Akses-web-browser-9870(Jihan)}
\caption{Akses melalui web browser dengan alamat http://localhost:9870}
\label{gam:Akses-web-browser-9870(Jihan)}
\end{figure} 


\end{enumerate}


\newday{\textbf{ September 2022}}
\begin{enumerate}
\item Kendala dan Solusi \\
% jelaskan kendala dan penyebab yang dialami saat mengikuti praktikum serta solusi atau langkah-langkah yang telah dilakukan 



\item Kesimpulan \\
% berikan kesimpulan dari praktikum yang telah dikerjkan


\end{enumerate}






\newthought{\textbf{M.IKHSAN - 2020903430016 - TRKJ 3B}}

\newday{\textbf{01 Desember 2022}}
\begin{enumerate}
\item Kendala
% jelaskan kendala dan penyebab yang dialami saat mengikuti praktikum serta solusi atau langkah-langkah yang telah dilakukan
\newline 
sudah berhasil menginstall hadoop

\item solusi
\newline
menginstal ulang ubuntu

\item Kesimpulan
% berikan kesimpulan dari praktikum yang telah dikerjkan
\newline
setelah menginstall ulang ubuntu, instalasi dan konfigurasi hadoop berhasil

\begin{figure}[!ht]
\includegraphics[width=\textwidth]{M.IKHSAN/jps}
\caption{Verifikasi Hasil Instalasi Hadoop}
\label{gam:Java-version(M.IKHSAN)}
\end{figure} 

\begin{figure}[!ht]
\includegraphics[width=\textwidth]{M.IKHSAN/localhost1}
\caption{Verifikasi Hasil Instalasi Hadoop}
\label{gam:Java-version(M.IKHSAN)}
\end{figure} 

\begin{figure}[!ht]
\includegraphics[width=\textwidth]{M.IKHSAN/localhost2}
\caption{Verifikasi Hasil Instalasi Hadoop}
\label{gam:Hadoop-version(M.IKHSAN)}
\end{figure}

\end{enumerate}


\newday{\textbf{02 Desember 2022}}
\begin{enumerate}
\item Kendala dan Solusi
% jelaskan kendala dan penyebab yang dialami saat mengikuti praktikum serta solusi atau langkah-langkah yang telah dilakukan

\item Kesimpulan
% berikan kesimpulan dari praktikum yang telah dikerjkan

\end{enumerate}

\newday{\textbf{08 Desember 2022}}
\begin{enumerate}
\item Kendala dan Solusi
% jelaskan kendala dan penyebab yang dialami saat mengikuti praktikum serta solusi atau langkah-langkah yang telah dilakukan

\item Kesimpulan
% berikan kesimpulan dari praktikum yang telah dikerjkan

\end{enumerate}


\newday{\textbf{09 Desember 2022}}
\begin{enumerate}
\item Kendala dan Solusi
% jelaskan kendala dan penyebab yang dialami saat mengikuti praktikum serta solusi atau langkah-langkah yang telah dilakukan


\item Kesimpulan
% berikan kesimpulan dari praktikum yang telah dikerjkan

\end{enumerate}

\newday{\textbf{15 Desember 2022}}
\begin{enumerate}
\item Kendala dan Solusi
\newline pada praktikum kali ini membuat program WordCount Bawaan Hadoop. pada saat praktikum tidak ada kendala hanya saja error karena salah menulis perintah.

\item Kesimpulan
Berhasil mencoba program bawaan Hadoop yaitu program menghitung jumlah kata dalam data input yang diberikan. Berikut ini gambar bukti keberhasilan praktikum.

\begin{figure}[!ht]
\includegraphics[width=\textwidth]{M.IKHSAN/wordcount bawaan1}
\caption{Cek Hasil}
\label{gam:Hadoop-version(M.IKHSAN)}
\end{figure}

\begin{figure}[!ht]
\includegraphics[width=\textwidth]{M.IKHSAN/WordCount}
\caption{Lihat Hasil}
\label{gam:Hadoop-version(M.IKHSAN)}
\end{figure}
\begin{figure}[!ht]
\includegraphics[width=\textwidth]{M.IKHSAN/WordCount2}
\caption{Lihat Hasil}
\label{gam:Hadoop-version(M.IKHSAN)}
\end{figure}
\end{enumerate}

\clearpage
\newday{\textbf{16 Desember 2022}}
\begin{enumerate}
\item Kendala
terdapat error pada perintah 6 dan 7

\item Solusi
- pada perintah no 6 saya memberi spasi antara JavaCompiled/ dengan WordCount.Java
- pada perintah no 7 saya kurang teliti di bagian JavaCompiled/ .


\item Kesimpulan
Untuk hasil yang ditampikan sama dengan WordCount bawaan hadoop.

\begin{figure}[!ht]
    \includegraphics[width=\textwidth]{M.IKHSAN/WordCount java.PNG}
    \caption{hasil WordCount Java }
    \label{gam:hasil}
    \end{figure}
\begin{figure}[!ht]
    \includegraphics[width=\textwidth]{M.IKHSAN/WordCount java1.PNG}
    \caption{hasil WordCount Java }
    \label{gam:hasil}
    \end{figure}
\begin{figure}[!ht]
    \includegraphics[width=\textwidth]{M.IKHSAN/WordCount java2.PNG}
    \caption{hasil WordCount Java}
    \label{gam:hasil}
    \end{figure}
\end{enumerate}

\newday{\textbf{22 Desember 2022}}
\begin{enumerate}
\item Kendala dan Solusi
% jelaskan kendala dan penyebab yang dialami saat mengikuti praktikum serta solusi atau langkah-langkah yang telah dilakukan

\begin{itemize}
\item Tidak menemukan masalah apapun
\end{itemize}

\item Kesimpulan
\newline
Apache Spark adalah sebuah framework komputasi yang dapat digunakan untuk mengakses data, memproses data, menanyakan data serta menganalisis big data

\begin{figure}[!ht]
\includegraphics[width=\textwidth]{M.IKHSAN/SPARK}
\caption{hasil instalasi apache spark }
\label{gam:hasil instalasi spark}
\end{figure}
\end{enumerate}

\newday{\textbf{23 Desember 2022} program wordcount dengan python}
\begin{enumerate}
\item Kendala  
1. kendala pertama ialah pada perintah nomor 6 terdapat error
2. kendala kedua ialah tidak bisa menjalankan perintah no 7. Terdapat pesan error 'Streaming Command Failed'

\item Solusi
1. solusinya adalah menambahkan python3 di dalam program map.py dan reduce.py karena versi python yang tersedia adalah python3, baru setelah itu program di lokal bisa di jalankan
2. saya sudah mencari kendalanya tapi tidak jumpa. saya juga sudah bertanya kepada rizki, tapi kendalanya tidak jumpa juga. jadi, saya memutuskan untuk install ulang ubuntu dan setelah saya instal ulang ubuntu ternyata berhasil.

\item Kesimpulan
% berikan kesimpulan dari praktikum yang telah dikerjkan
Berhasil menjalankan program wordcount dengan python walaupun banyak mengalami kendala, berikut bukti hasil praktikum. 

\begin{figure}[!ht]
    \includegraphics[width=\textwidth]{M.IKHSAN/ResultWordCountPython}
    \caption{cek hasil }
    \label{gam:hasil}
    \end{figure}
\begin{figure}[!ht]
    \includegraphics[width=\textwidth]{M.IKHSAN/ResultWordCountPython1}
    \caption{lihat hasil}
    \label{gam:hasil}
    \end{figure}
\begin{figure}[!ht]
    \includegraphics[width=\textwidth]{M.IKHSAN/ResultWordCountPython2}
    \caption{lihat hasil}
    \label{gam:hasil}
    \end{figure}
\end{enumerate}

\newday{\textbf{02 Januari 2023} Wordcountpyspark}
\begin{enumerate}
\item Kendala dan Solusi
\item tidak menemukan masalah pada pratikum

\item Kesimpulan
Berhasil menjalankan program menggunakan PySpark

\begin{figure}[!ht]
    \includegraphics[width=\textwidth]{M.IKHSAN/wordcountpyspark}
    \caption{hasil program WordCountPySpark }
    \label{gam:hasil instalasi spark}
    \end{figure}
\begin{figure}[!ht]
    \includegraphics[width=\textwidth]{M.IKHSAN/wordcountpyspark1}
    \caption{hasil program WordCountPySpark }
    \label{gam:hasil instalasi spark}
    \end{figure}
\begin{figure}[!ht]
    \includegraphics[width=\textwidth]{M.IKHSAN/wordcountpyspark2}
    \caption{hasil program WordCountPySpark }
    \label{gam:hasil instalasi spark}
    \end{figure}
\end{enumerate}
\include{laporan/MuhammadIkrammullah}
\newthought{\textbf{Muhammad Munawir - 2020903430026 - TRKJ 3B}}

\newday{\textbf{1 - 2 Desember 2022} - Instalasi dan Konfigurasi Apache Hadoop}
\begin{enumerate}
\item Kendala dan Solusi \\
Kendala :\\
Saat melakukan download hadoop, terkendala dengan space disk virtual box yang tidak cukup.
Saat melakukan konfigurasi apache hadoop, tidak dapat menjalankan format-hdfs. 

Solusi :\\
Karena virtual box tidak bisa menambahkan kapasitas space seperti di VMware. Maka melakukan instalasi ulang ubuntu. Kemudian download hadoop berhasil.
Karena tidak dapat menjalankan format-hdfs, jadi mencoba konfigurasi ulang. dan hasilnya berhasil.
dan saat menjalankan test jps, hasilnya ada 5 keluaran.

\item Kesimpulan \\
Praktikum penginstalan hadoop dan konfigurasi hadoop berhasil dijalankan sesuai perintah-perintah yang ada.

\begin{figure}[!ht]
    \includegraphics[width=\textwidth]{Munawir/InstalasiHadoop}
    \caption{hasil dari cek versi hadoop}
    \label{gam:perkuliahan-25-11}
    \end{figure}

\begin{figure}[!ht]
    \includegraphics[width=\textwidth]{Munawir/localhost9870}
    \caption{hasil dari localhost:9870}
    \label{gam:perkuliahan-25-11}
    \end{figure}

\begin{figure}[!ht]
    \includegraphics[width=\textwidth]{Munawir/jps}
    \caption{hasil dari menjalankan jps}
    \label{gam:perkuliahan-25-11}
    \end{figure}

\begin{figure}[!ht]
    \includegraphics[width=\textwidth]{Munawir/localhost8088}
    \caption{hasil dari localhost:8088}
    \label{gam:perkuliahan-25-11}
    \end{figure}

\end{enumerate}

% END . . . . . . . . . . . . . .

\newday{\textbf{8 Desember 2022}}
\begin{enumerate}
\item Kendala dan Solusi
% jelaskan kendala dan penyebab yang dialami saat mengikuti praktikum serta solusi atau langkah-langkah yang telah dilakukan

\item Kesimpulan
% berikan kesimpulan dari praktikum yang telah dikerjkan

\end{enumerate}

\newday{\textbf{9 Desember 2022}}
\begin{enumerate}
\item Kendala dan Solusi
% jelaskan kendala dan penyebab yang dialami saat mengikuti praktikum serta solusi atau langkah-langkah yang telah dilakukan

\item Kesimpulan
% berikan kesimpulan dari praktikum yang telah dikerjkan

\end{enumerate}

\newday{\textbf{15 Desember 2022} WordCount Hadoop}
\begin{enumerate}
\item Kendala dan Solusi
% jelaskan kendala dan penyebab yang dialami saat mengikuti praktikum serta solusi atau langkah-langkah yang telah dilakukan
Saat mencoba melakukan perintah/program WordCount bawaan hadoop, langakah ke 5. dimana muncul pemberitahuan untuk {\textbf{"apt install emboss"}}. setelah mencoba menjalankan perintah "sudo apt install emboss" dan mencoba konfigurasi ulang WordCount. Program WordCount dapat berjalan.

\begin{figure}[!ht]
    \includegraphics[width=\textwidth]{Munawir/wordcountHadoop}
    \caption{hasil dari WordCount bawaan hadoop}
    \label{gam:perkuliahan-25-11}
    \end{figure}

\item Kesimpulan
% berikan kesimpulan dari praktikum yang telah dikerjkan
Program WordCount bawaan hadoop atau hadoop itu sendiri sedikit sensitif, soalnya sedikit kesalahan Program atau aplikasi pendukung yang kurang. Maka WordCount hadoop tidak dapat dijalankan.
\end{enumerate}

\newday{\textbf{16 Desember 2022}WordCount Java}
\begin{enumerate}
\item Kendala dan Solusi
% jelaskan kendala dan penyebab yang dialami saat mengikuti praktikum serta solusi atau langkah-langkah yang telah dilakukan
Kendala dan Solusi Pada pertemuan hari ini, kegiatan yang dilakukan adalah mencoba program wordcount dengan java. Selama praktikum mengalami kendala pada poin ke-6 berdasarkan urutan di modul. Program tidak mau dicompile karena kesalahan penulisan perintah.

\begin{figure}[!ht]
    \includegraphics[width=\textwidth]{Munawir/wordcountJava}
    \caption{hasil dari WordCount dengan Program Java}
    \label{gam:perkuliahan-25-11}
    \end{figure}

\item Kesimpulan
% berikan kesimpulan dari praktikum yang telah dikerjkan
Dapat memberikan pemahaman dasar mengenai proses program WordCount java, dan meng-compile Program hingga menjalankan program.

\end{enumerate}

\newday{\textbf{22 Desember 2022}Instalasi Apache Spark (PySpark)}
\begin{enumerate}
\item Kendala dan Solusi
Tidak ada kendala saat proses intalasi

\item Kesimpulan
Proses penginstalan Apache Spark berhasil dilakukan tanpa
adanya kendala satupun

\begin{figure}[!ht]
    \includegraphics[width=\textwidth]{Munawir/pysparkVersion}
    \caption{hasil dari instalasi PySpark}
    \label{gam:perkuliahan-25-11}
    \end{figure}

\end{enumerate}
\newthought{\textbf{Nadzura Kumaira - 2020903430030 - TRKJ 3B}}

\newday{\textbf{1 Desember 2022}- Instalasi Apache Hadoop}
\begin{enumerate}
\item Kendala dan Solusi
\newline berada di perintah hadoop version dimana ketika dilakukan perintah itu tidak keluar,tapi saya membuat ulang dan akhirnya bisa

\item Kesimpulan
\newline jangan ada perintah yang ketinggalan karena itu membuat error.

\begin{figure}
\includegraphics[width=\textwidth]
{NadzuraKumaira/hadoop version}
\caption{hasil dari hadoop version}
\label{gam:perkuliahan-25-11}
\end{figure}

\begin{figure}
\includegraphics[width=\textwidth]
{NadzuraKumaira/jps}
\caption{chek jps}
\label{gam:perkuliahan-25-11}
\end{figure}
\end{enumerate}

\newday{\textbf{2 desember 2022}konfigurasi apache hadoop}
\begin{enumerate}
\item Kendala dan Solusi
\newline jangan lupa sertakan <configuration> dan </configuration> setiap membuat sudo nano nama-file.
\item Kesimpulan
\newline di perhatikan tanda petik jangan ada yang terbalik.

\begin{figure}
\includegraphics[width=\textwidth]
{NadzuraKumaira/localhost}
\caption{hasil dari localhost} 
\label{gam:perkuliahan-20-11}
\end{figure}

\begin{figure}
\includegraphics[width=\textwidth]
{NadzuraKumaira/localhost888}
\caption{hasil dari localhost888}
\label{gam:perkuliahan-20-11}
\end{figure}
\end{enumerate}

\newday{\textbf{8 desember 2022}}
\begin{enumerate}
\item Kendala dan Solusi
\item Kesimpulan
\end{enumerate}

\newday{\textbf{9 desember 2022}}
\begin{enumerate}
\item Kendala dan Solusi
\item Kesimpulan
\end{enumerate}

\newday{\textbf{15 desember 2022}program wordcount bawaan hadoop}
\begin{enumerate}
\item Kendala dan Solusi
\newline poin ke-3 jangan lupa tambah kata sudo nano.
\item Kesimpulan
\newline jangan ada perintah yang ketinggalan.

\begin{figure}
\includegraphics[width=\textwidth]
{NadzuraKumaira/hasilresultwordcount}
\caption{hasil dari hasilresultwordcount}
\label{gam:perkuliahan-10-11}
\end{figure}
\end{enumerate}

\newday{\textbf{16 desember 2022}program wordcount dengan java}
\begin{enumerate}
\item Kendala dan Solusi
\newline jangan ada perintah yang ketinggalan atau akan pusing sendiri.
\item Kesimpulan
\newline berusaha jangan mudah menyerah.

\begin{figure}
\includegraphics[width=\textwidth]
{NadzuraKumaira/hasilresultwordcount}
\caption{hasil dari hasilresultwordcount}
\label{gam:perkuliahan-17-11}
\end{figure}
\end{enumerate}

\newday{\textbf{22 desember 2022}instalasi apache spark(pyspark)}
\begin{enumerate}
\item Kendala dan Solusi
\newline jaringan wifi harus bagus agar berhasil
\item Kesimpulan
\newline jangan panik kalau macet.

\begin{figure}
\includegraphics[width=\textwidth]
{NadzuraKumaira/pyspark}
\caption{hasil dari pyspark}
\label{gam:perkuliahan-24-11}
\end{figure}

\begin{figure}
\includegraphics[width=\textwidth]
{NadzuraKumaira/hasilpyspark}
\caption{hasil dari hasilpyspark}
\label{gam:perkuliahan-24-11}
\end{figure}
\end{enumerate}

\newday{\textbf{23 desember 2022}program wordcound dengan python}
\begin{enumerate}
\item Kendala dan Solusi
\newline di perhatikan tanda petiknya.
\item Kesimpulan
\newline tanda petik bikin error.

\begin{figure}
\includegraphics[width=\textwidth]
{NadzuraKumaira/pyspark}
\caption{hasil dari pyspark}
\label{gam:perkuliahan-8-12}
\end{figure}

\begin{figure}
\includegraphics[width=\textwidth]
{NadzuraKumaira/hasilpyspark}
\caption{hasil dari hasilpyspark}
\label{gam:perkuliahan-8-12}
\end{figure}
\end{enumerate}

\newday{\textbf{15 desember 2022}tugas individu}
\begin{enumerate}
\item Kendala dan Solusi
\newline di perhatikan tanda petiknya.
\item Kesimpulan
\newline tanda petik bikin error.

\begin{figure}
\includegraphics[width=\textwidth]
{NadzuraKumaira/tabel1}
\caption{hasil dari tabel1}
\caption{hasil dari tabel1}
\label{gam:perkuliahan-15-12}
\end{figure}

\begin{figure}
\includegraphics[width=\textwidth]
{NadzuraKumaira/tabel2}
\caption{hasil dari tabel2}
\label{gam:perkuliahan-15-12}
\end{figure}

\begin{figure}
\includegraphics[width=\textwidth]
{NadzuraKumaira/grapik1}
\caption{hasil dari grapik1}
\label{gam:perkuliahan-15-12}
\end{figure}

\begin{figure}
\includegraphics[width=\textwidth]
{NadzuraKumaira/grapik2}
\caption{hasil dari grapik2}
\label{gam:perkuliahan-15-12}
\end{figure}
\end{enumerate}

\newthought{\textbf{Nurani Harum Fardaniah - 2020903430034 - TRKJ 3B}}

\newday{\textbf{1 Desember 2022}}
\begin{enumerate}
\item Kendala dan Solusi
% jelaskan kendala dan penyebab yang dialami saat mengikuti praktikum serta solusi atau langkah-langkah yang telah dilakukan

\item Kesimpulan
% berikan kesimpulan dari praktikum yang telah dikerjkan

\end{enumerate}

\newday{\textbf{2 Desember 2022}}
\begin{enumerate}
\item Kendala dan Solusi
% jelaskan kendala dan penyebab yang dialami saat mengikuti praktikum serta solusi atau langkah-langkah yang telah dilakukan

\item Kesimpulan
% berikan kesimpulan dari praktikum yang telah dikerjkan

\end{enumerate}

\newday{\textbf{8 Desember 2022}}
\begin{enumerate}
\item Kendala dan Solusi
% jelaskan kendala dan penyebab yang dialami saat mengikuti praktikum serta solusi atau langkah-langkah yang telah dilakukan

\item Kesimpulan
% berikan kesimpulan dari praktikum yang telah dikerjkan

\end{enumerate}

\newthought{\textbf{Nuraula Tafiza - 2020903430035 - TRKJ 3B}}

\newday{\textbf{22 September 2022}}
\begin{enumerate}
\item Kendala dan Solusi
\\ Tidak ada kendala
%jelaskan kendala dan penyebab yang dialami saat mengikuti praktikum serta solusi atau langkah-langkah yang telah dilakukan

\item Kesimpulan
\\ Berhasil melakukan Instalasi.
% berikan kesimpulan dari praktikum yang telah dikerjkan

\end{enumerate}
\newthought{\textbf{Nurul Aflah - 2020903430036 - TRKJ 3B}}

\newday{\textbf{15 Desember 2022}}
\begin{enumerate}
\item Kendala dan Solusi
% jelaskan kendala dan penyebab yang dialami saat mengikuti praktikum serta solusi atau langkah-langkah yang telah dilakukan

\item Kesimpulan
% berikan kesimpulan dari praktikum yang telah dikerjkan

\end{enumerate}


\newthought{\textbf{Rauzatinur Syah - 2020903430039 - TRKJ 3B}}


\newday{\textbf{1 - 2 Desember 2022} - Instalasi dan Konfigurasi Hadoop}
\begin{enumerate}
\item Kendala dan Solusi\\
% jelaskan kendala dan penyebab yang dialami saat mengikuti praktikum serta solusi atau langkah-langkah yang telah dilakukan
\begin{enumerate}
\item Kendala
\begin{itemize}
\item terdapat kendala pada instalasi hadoop pada pengecekanan versi hadoop yang berfungsi untuk menverifikasi instalasi hadoop
\item terdapat kendala saat menjalanakn format HDFS dan hadoop service
\end{itemize}
\item Solusi \\
\begin{itemize}
\item melakukan pengecekan pada file hadoop-env.sh
\item melakuka  pengecekan pada file core-site.xml, hdfs-site.xml, mapred-site.xml, yarn-site.xml
\end{itemize}
\end{enumerate}

\item Kesimpulan\\
% berikan kesimpulan dari praktikum yang telah dikerjkan
adapun kesimpulan yang diperoleh yaitu instalasi dan konfigurasi hadoop berhasil 
\end{enumerate}

\begin{figure}[!ht]
\includegraphics[width=\textwidth]{RauzatinurSyah/dataHadoop}
\caption{hasil program WordCount hadoop}
\label{gam:Hasil}
\end{figure}

\newday{\textbf{8 Desember 2022} - WordCount bawaan Hadoop}
\begin{enumerate}
\item Kendala dan Solusi\\
% jelaskan kendala dan penyebab yang dialami saat mengikuti praktikum serta solusi atau langkah-langkah yang telah dilakukan
pada pratikum Program WordCount bawaan hadoop tidak ada kendala pada praktikum yang dilakukan
\item Kesimpulan\\
% berikan kesimpulan dari praktikum yang telah dikerjkan
berhasil menjalanakan program wordCount bawan hadoop
\end{enumerate}


\begin{figure}[!ht]
\includegraphics[width=\textwidth]{RauzatinurSyah/datahadoopjava no 9}
\caption{hasil program WordCount java no.9}
\label{gam:hasil program}
\end{figure}

\begin{figure}[!ht]
\includegraphics[width=\textwidth]{RauzatinurSyah/datahadoopjava no 10}
\caption{hasil program WordCount java no.10}
\label{gam:hasil program}
\end{figure}

\newday{\textbf{9 Desember 2022} - WordCount dengan Java}
\begin{enumerate}
\item Kendala dan Solusi \\
% jelaskan kendala dan penyebab yang dialami saat mengikuti praktikum serta solusi atau langkah-langkah yang telah dilakukan
pada praktikum WordCount dengan java
\begin{enumerate}
\item kendala: \\
terdapat kendala pada perintah menjalankan program dengan perintah"hadoop jar WordCount.jar WordCount /input/data/WordCount.txt /ResultWourdCountJava"
\item solusi: \\
menjalankan hadoop service dengan perintah "start-all.sh" dikarnakan terdapat kesalahan yang dilakukan yaitu tidak menjalankan hadoop service
\end{enumerate}
\item Kesimpulan\\
% berikan kesimpulan dari praktikum yang telah dikerjkan
adapun kesimpulan yang diperoleh yaitu berhasil menjalankan programa tersebut 
\end{enumerate}

\newday{\textbf{15 Desember 2022} - instalasi apache spark}
\begin{enumerate}
\item Kendala dan Solusi\\
% jelaskan kendala dan penyebab yang dialami saat mengikuti praktikum serta solusi atau langkah-langkah yang telah dilakukan
pada instalasi apache spark tidak ada kendala yang dialami 
\item Kesimpulan\\
% berikan kesimpulan dari praktikum yang telah dikerjkan
apache spark berhasil dijalankan

\end{enumerate}

\begin{figure}[!ht]
\includegraphics[width=\textwidth]{RauzatinurSyah/install spark}
\caption{hasil instalasi apache spark }
\label{gam:hasil instalasi spark}
\end{figure}


\newday{\textbf{16 Desember 2022} - WordCount Dengan Python}
\begin{enumerate}
\item Kendala dan Solusi\\
% jelaskan kendala dan penyebab yang dialami saat mengikuti praktikum serta solusi atau langkah-langkah yang telah dilakukan
\begin{itemize}
\item terdapat kendala pada saat melakukan percobaan program dilocal, yang mana hasil percobaan tersebut tidak muncul
\item solusi yang dilakukan yaitu melakukan pengecekan ulang pada file map.py dan reduce.py

\end{itemize}

\item Kesimpulan\\
% berikan kesimpulan dari praktikum yang telah dikerjkan
adapaun kesimpulan dari praktikum ini yaitu program yang dijalankan berhasil, dengan bukti screenshot sebagai berikut

\begin{figure}[!ht]
\includegraphics[width=\textwidth]{RauzatinurSyah/no8 wordcount python}
\caption{hasil langkah ke-8 }
\label{gam:hasil program WordCountPython}
\end{figure}

\begin{figure}[!ht]
\includegraphics[width=\textwidth]{RauzatinurSyah/no9 wordcount python}
\caption{hasil langkah ke-9 }
\label{gam:hasil program WordCountPython}
\end{figure}

\end{enumerate}


\newday{\textbf{22 Desember 2022} - WordCount dengan pySpark}
\begin{enumerate}
\item Kendala dan Solusi
% jelaskan kendala dan penyebab yang dialami saat mengikuti praktikum serta solusi atau langkah-langkah yang telah dilakukan
\begin{itemize}

\item adapun kendalam pada praktikum ini terdapat pada langkah ke 6, yang dilakukan pengecekan hasil dengan printah "hadoop fs -ls /ResultWordCountPyspark" terdapat error yang menyatakan bahwa direktori tersebut tidak ada.

\item solusi yang diatasi pada kendala tersebut yaitu menulis kembali perintah "hadoop fs -ls /ResultWordCountPySpark" dengan "PySpark". perintah awal akan error dikarnakan direktori yang dibuat yaitu WOrdcountPySpark dengan kata spark diawali huruf kapital.
\end{itemize}

\item Kesimpulan\\
% berikan kesimpulan dari praktikum yang telah dikerjkan
adapun kesimpulan yang diperoleh yaitu program WordCount dengan PySpark berhasil dijalankan, dengan hasil screentshot sebagai berikut:

\begin{figure}[!ht]
\includegraphics[width=\textwidth]{RauzatinurSyah/no6 wordcount Pyspark}
\caption{hasil langkah ke-6 }
\label{gam:hasil program WordCountPySpark}
\end{figure}

\begin{figure}[!ht]
\includegraphics[width=\textwidth]{RauzatinurSyah/no7 wordcount Pyspark}
\caption{hasil langkah ke-7 }
\label{gam:hasil program WordCountPySpark}
\end{figure}
\end{enumerate}

\newday{\textbf{23 Desember 2022} - Tugas Individu}
\begin{enumerate}
\item Kendala dan Solusi
% jelaskan kendala dan penyebab yang dialami saat mengikuti praktikum serta solusi atau langkah-langkah yang telah dilakukan
\begin{itemize}
\item kendala yang terdapat pada tugas ini yaitu kesalah ketika menulis Kmeans pada sintak for dan kesalahan ketika menulis KMeans Assignments pada cluster assignment yang menyatakan bahwa KMeans Assignments tidak terdeskripsi
\item soslusi yang diberikan yaitu penulisan KMeans pada for ditulis dengan mengetab tombol tab pada keyboard 1 kali. adapun penulisannya adalah sebagai berikut

\newpage
\begin{figure}[!ht]
\includegraphics[width=\textwidth]{RauzatinurSyah/penulisan dalam for}
\caption{penulisan dalam for}
\label{gam:penulisan dalam for}
\end{figure}

dan untuk KMeans Assignemnt seharusnya tidak terdapat kesalah apabila langkah ke 6 tidak terloncati

\end{itemize}

\item Kesimpulan
% berikan kesimpulan dari praktikum yang telah dikerjkan

adapun kesimpulan pada tugas ini yaitu seluaruh program yang terdapat pada module dapat dijalankan dan mengahsilkan bebrapa table data dana figure. adapaun hasil yang diperoleh adalah sebagai berikut:

\begin{figure}[!ht]
\includegraphics[width=\textwidth]{RauzatinurSyah/Tgs-individu no 3_4}
\caption{tampilan table 1}
\label{gam:hasil tugas individu}
\end{figure}

\begin{figure}[!ht]
\includegraphics[width=\textwidth]{RauzatinurSyah/Tgs-individu no 5 assembled_datashow}
\caption{tampilan table 2}
\label{gam:hasil tugas individu}
\end{figure}

\begin{figure}[!ht]
\includegraphics[width=\textwidth]{RauzatinurSyah/Tgs-individu no 5 figure}
\caption{tampilan figure 1}
\label{gam:hasil tugas individu}
\end{figure}

\begin{figure}[!ht]
\includegraphics[width=\textwidth]{RauzatinurSyah/Tgs-individu no 6 figure}
\caption{tampilan figure 2}
\label{gam:hasil tugas individu}
\end{figure}

\end{enumerate}




\newthought{\textbf{Resha Russita - 2020903430040 - TRKJ 3B}}

\newday{\textbf{22 September 2022}}
\begin{enumerate}
\item Kendala dan Solusi
\newline Pada Instalasi Apache Hadoop, saya sebagai praktikan mengalami kendala saat mengekstrak file Apache Hadoop. Solusi yang saya gunakan adalah membuat kembali os ubuntu dengan ukuran ruang yang lebih besar dari sebelumnya sehingga file yang diekstrak berhasil. 
Kendala lain yang terjadi adalah hanya kurang teliti saat melakukan konfigurasi pada Apache Hadoop.

\begin{figure}[!ht]
\includegraphics[width=\textwidth]{ReshaRussita/jpshadoopservice-resha}
\caption{cek services hadoop dengan jps}
\label{gam:perkuliahan-22-09}
\end{figure}

\begin{figure}[!ht]
\includegraphics[width=\textwidth]{ReshaRussita/localhost8088-resha}
\caption{local host 8088 akses}
\label{gam:perkuliahan-22-09}
\end{figure}

\begin{figure}[!ht]
\includegraphics[width=\textwidth]{ReshaRussita/localhost9870-resha}
\caption{local host 9870 akses}
\label{gam:perkuliahan-22-09}
\end{figure}

\newpage
\item Kesimpulan
\newline Pada Instalasi Apache Hadoop membutuhkan ruang yang cukup besar untuk mengekstrak file Apache Hadoop. Penginstalaan Apache Hadoop harus dilakukan sesuai step, begitu juga pada saat konfigurasi. Konfigurasi ini bertujuan untuk memudahkan user dalam memonitoring ekosistem di dalam Hadoop. Saat mengkonfigurasi terdapat perintah untuk membuat format HDFS yang berfungsi menyimpan suatu data dengan cara membaginya menjadi potong-potongan data yang disebut blok berukuran 64 MB dan kemudian disimpan pada node-node yang tersebar dalam kluster. Node-node yang ada adalah name node dan data node. Sehingga saat mengecek Hadoop service dengan perintah jps, kedua node tersebut harus tersedia sebagai pendukung saat dilakukan akses web browser dengan local host 9870 dan 8088 itu berhasil.

\end{enumerate}

\newday{\textbf{24 November 2022}}
\begin{enumerate}
\item Kendala dan Solusi
\newline Pada proses instalasi git dan konfigurasi git dengan github, saya sebagai praktikan tidak mengalami kendala sama sekali.

\item Kesimpulan
\newline Pada instalasi git dan konfigurasi git dengan github bertujuan untuk menjalankan perintah-perintah git melalui penyimpan local dan menyimpan hasil pekerjaan pada github. Penyimpanan pekerjaan dapat dilakukan dengan mudah pada github dengan hanya melakukan git push.

\end{enumerate}

\newday{\textbf{01 Desember 2022}}
\begin{enumerate}
\item Kendala dan Solusi
\newline Pada program WordCount bawaan Hadoop, tidak ada kendala yang saya alami.

\begin{figure}[!ht]
\includegraphics[width=\textwidth]{ReshaRussita/langkah6dan7-resha}
\caption{Hasil perhitungan WordCount bawaan Hadoop berdasarkan data output}
\label{gam:perkuliahan-08-12}
\end{figure}

\item Kesimpulan
\newline Pada Hadoop terdapat program untuk menghitung jumlah kata (WordCount) yang ada pada data. Untuk menghitung jumlah kata, saya sebagai praktikan melakukan input data terlebih dahulu, kemudian memprosesnya, sehingga menghasilkan data output. Data output tersebut yang digunakan untuk Hadoop menjalankan programnya yaitu WordCount.
Data yang saya input disini adalah data mahasiswa (nama, asal sekolah, dan alamat). Saat melihat hasil perhitungan pada data output, akan ditampilkan jumlah kata dari tiap-tiap nama, asal sekolah, dan alamat mahasiswa.

\end{enumerate}

\newday{\textbf{02 Desember 2022}}
\begin{enumerate}
\item Kendala dan Solusi
\newline Pada program WordCount bawaan Java, tidak ada kendala yang saya alami.

\begin{figure}[!ht]
\includegraphics[width=\textwidth]{ReshaRussita/langkah9dan10-resha}
\caption{Hasil perhitungan WordCount dengan Java berdasarkan data output}
\label{gam:perkuliahan-08-12}
\end{figure}

\item Kesimpulan
\newline Praktikum ini yaitu menghitung jumlah kata (WordCount) yang ada pada data Java. Proses yang dijalankan mulai dari membuat program dan menyiapkan data untuk WordCount java, kemudian meng-compile program. Program yang dihasilkan sama seperti yang ditampilkan pada WordCount bawaan Hadoop.

\end{enumerate}

\newday{\textbf{08 Desember 2022}}
\begin{enumerate}
\item Kendala dan Solusi
% jelaskan kendala dan penyebab yang dialami saat mengikuti praktikum serta solusi atau langkah-langkah yang telah dilakukan

\item Kesimpulan
% berikan kesimpulan dari praktikum yang telah dikerjkan

\end{enumerate}

\newday{\textbf{09 Desember 2022}}
\begin{enumerate}
\item Kendala dan Solusi
% jelaskan kendala dan penyebab yang dialami saat mengikuti praktikum serta solusi atau langkah-langkah yang telah dilakukan

\item Kesimpulan
% berikan kesimpulan dari praktikum yang telah dikerjkan

\end{enumerate}
\newthought{\textbf{Rizki Ilhami - 2020903430042 - TRKJ 3B}}

\newday{\textbf{1 Desember 2022}}
\begin{enumerate}
\item Kendala dan Solusi
% jelaskan kendala dan penyebab yang dialami saat mengikuti praktikum serta solusi atau langkah-langkah yang telah dilakukan

\item Kesimpulan
% berikan kesimpulan dari praktikum yang telah dikerjkan

\end{enumerate}

\newday{\textbf{2 Desember 2022}}
\begin{enumerate}
\item Kendala dan Solusi
% jelaskan kendala dan penyebab yang dialami saat mengikuti praktikum serta solusi atau langkah-langkah yang telah dilakukan

\item Kesimpulan
% berikan kesimpulan dari praktikum yang telah dikerjkan

\end{enumerate}

\newday{\textbf{8 Desember 2022} - Instalasi Hadoop}
\begin{enumerate}
\item Kendala dan Solusi

\begin{itemize}
\item Tidak menumukan kendala apapun.
\end{itemize}

\item Kesimpulan
% berikan kesimpulan dari praktikum yang telah dikerjkan

\end{enumerate}

\newday{\textbf{9 Desember 2022} -}
\begin{enumerate}
\item Kendala dan Solusi
% jelaskan kendala dan penyebab yang dialami saat mengikuti praktikum serta solusi atau langkah-langkah yang telah dilakukan

\item Kesimpulan
% berikan kesimpulan dari praktikum yang telah dikerjkan

\end{enumerate}

\newday{\textbf{15 Desember 2022} - instalasi dan konfigurasi hadoop}
\begin{enumerate}
\item Kendala dan Solusi

\begin{itemize}
\item Tidak dapat membuka browser Firefox
\end{itemize}


\item Kesimpulan
% berikan kesimpulan dari praktikum yang telah dikerjkan

\end{enumerate}

\newday{\textbf{16 Desember 2022}}
\begin{enumerate}
\item Kendala dan Solusi

\begin{itemize}
\item Tidak dapat membuka browser Firefox
\end{itemize}


\item Kesimpulan
% berikan kesimpulan dari praktikum yang telah dikerjkan

\end{enumerate}

\newday{\textbf{22 Desember 2022}}
\begin{enumerate}
\item Kendala dan Solusi

\begin{itemize}
\item
\end{itemize}


\item Kesimpulan
% berikan kesimpulan dari praktikum yang telah dikerjkan

\end{enumerate}
	
\newthought{\textbf{Salsabila Irmanda - 2020903430048 - TRKJ 3B}}

\newday{\textbf{1 - 2 Desember 2022} - Instalasi dan Konfigurasi Hadoop}
\begin{enumerate}
\item Kendala dan Solusi\\
\begin{enumerate}
\item Saat melakukan percobaan instalasi hadoop terdapat kendala yaitu saat melakukan extrak file hadoop, solusi yang digunakan yaitu memberikan ukuran ruang lebih besar saat membuat os ubuntu
\item Melakukan konfigurasi apache hadoop pratikan mengalami kendala pada perintah format HDFS. solusi yang digunakan yaitu mengecek lagi file - file konfigurasi hadoop hingga tidak ada lagi kesalahan
\end{enumerate}
% jelaskan kendala dan penyebab yang dialami saat mengikuti praktikum serta solusi atau langkah-langkah yang telah dilakukan

\item Kesimpulan \\
Adapun kesimpulan yang diperoleh yaitu tahap instalasi dan konfigurasi hadoop telah berhasil. untuk mengecek hadoop service dengan perintah jps.
% berikan kesimpulan dari praktikum yang telah dikerjkan
\begin{figure}[!ht]
\includegraphics[width=.8\textwidth]{SalsabilaIrmanda/1}
\caption{localhost:8088}
\label{gam:perkuliahan-22-09}
\end{figure}

\newpage
\begin{figure}[!ht]
\includegraphics[width=\textwidth]{SalsabilaIrmanda/2}
\caption{localhost:9870}
\label{gam:perkuliahan-22-09}
\end{figure}
\end{enumerate}


\newday{\textbf{8 Desember 2022} - WordCount bawaan Hadoop}
\begin{enumerate}
\item Kendala dan Solusi \\
Saat melakukan percobaan program Word Count Hadoop tidak mengalami kendala. 
% jelaskan kendala dan penyebab yang dialami saat mengikuti praktikum serta solusi atau langkah-langkah yang telah dilakukan
\item Kesimpulan \\
setelah melanjutkan percobaan dan mengikuti semua langkah hingga selesai pratikan berhasil menjalankan ouput program wordcount bawaan hadoop. 

% berikan kesimpulan dari praktikum yang telah dikerjkan
\begin{figure}[!ht]
\includegraphics[width=\textwidth]{SalsabilaIrmanda/langkah6&7}
\caption{Hasil perhitungan Wordcount bawaan hadoop berdasarkan output}
\label{gam:perkuliahan-08-12}
\end{figure}
\end{enumerate}

\newday{\textbf{9 Desember 2022} - WordCount dengan Java}
\begin{enumerate}
\item Kendala dan Solusi \\
Terdapat kendala saat melakukan cek hasil pada program wordcount dengan java, solusi nya yaitu harus menjalankan hadoop service. 
% jelaskan kendala dan penyebab yang dialami saat mengikuti praktikum serta solusi atau langkah-langkah yang telah dilakukan

\item Kesimpulan \\
percobaan yang telah dilakukan yaitu melakukan proses membuat program, menyiapkan data, meng-compile program hingga menjalankan program dan berhasil menampilkan hasilnya.
% berikan kesimpulan dari praktikum yang telah dikerjkan

\begin{figure}[!ht]
\includegraphics[width=\textwidth]{SalsabilaIrmanda/langkah9}
\caption{hasil program wordcount java 9}
\label{gam:hasil}
\end{figure}

\begin{figure}[!ht]
\includegraphics[width=\textwidth]{SalsabilaIrmanda/langkah10}
\caption{hasil program wordcount java 10}
\label{gam:hasil}
\end{figure}
\end{enumerate}

\newpage
\newday{\textbf{15 Desember 2022}}
\begin{enumerate}
\item Kendala dan Solusi
\newline pada percobaan instalasi apache spark, saat menjalankan perintah verifikasi hasil intalasi tidak muncul. solusi yang digunakan yaitu mengecek kembali program pada langkah ke 4.
% jelaskan kendala dan penyebab yang dialami saat mengikuti praktikum serta solusi atau langkah-langkah yang telah dilakukan

\item Kesimpulan
setelah melakukan pengecekan hingga tidak terdapat lagi kesalahan maka pyspark berhasil dijalankan.
% berikan kesimpulan dari praktikum yang telah dikerjkan
\begin{figure}[!ht]
\includegraphics[width=\textwidth]{SalsabilaIrmanda/installspark}
\caption{hasil verifikasi instalasi spark}
\label{gam:hasil}
\end{figure}
\end{enumerate}

\newday{\textbf{16 Desember 2022}}
\begin{enumerate}
\item Kendala dan Solusi
% jelaskan kendala dan penyebab yang dialami saat mengikuti praktikum serta solusi atau langkah-langkah yang telah dilakukan

\item Kesimpulan
% berikan kesimpulan dari praktikum yang telah dikerjkan

\end{enumerate}
\newthought{\textbf{Siti Hajar Al Zahra - 2020903430049 - TRKJ 3B}}

\newday{\textbf{1 - 2 Desember 2022} Instalasi dan Konfigurasi Apache Hadoop}
\begin{enumerate}
\item Kendala dan Solusi
\newline pada praktikum instalasi apache hadoop tidak terdapat kendala 
% jelaskan kendala dan penyebab yang dialami saat mengikuti praktikum serta solusi atau langkah-langkah yang telah dilakukan

\item Kesimpulan
\newline berhasil melakukan instalasi dan konfigurasi hadoop.
% berikan kesimpulan dari praktikum yang telah dikerjkan
\begin{figure}[!ht]
\includegraphics[width=\textwidth]{SitiHajarAlZahara/1}
\caption{hasil instalasi java}
\label{gam:Hasil}
\end{figure}

\begin{figure}[!ht]
\includegraphics[width=\textwidth]{SitiHajarAlZahara/2}
\caption{hasil konfigurasi apache hadoop}
\label{gam:Hasil}
\end{figure}

\begin{figure}[!ht]
\includegraphics[width=\textwidth]{SitiHajarAlZahara/3}
\caption{hasil konfigurasi apache hadoop}
\label{gam:Hasil}
\end{figure}


\end{enumerate}

\newday{\textbf{8 - 9 Desember 2022} WordCount bawaan Hadoop}
\begin{enumerate}
\item Kendala dan Solusi
\newline pada praktikum wordcount ini pratikan tidak terdapat kendala 
% jelaskan kendala dan penyebab yang dialami saat mengikuti praktikum serta solusi atau langkah-langkah yang telah dilakukan

\item Kesimpulan
\newline berhasil melakukan konfigurasi WordCount Hadoop.
% berikan kesimpulan dari praktikum yang telah dikerjkan
\begin{figure}[!ht]
\includegraphics[width=\textwidth]{SitiHajarAlZahara/4}
\caption{ folder input di HDFS}
\label{gam:Hasil}
\end{figure}

\begin{figure}[!ht]
\includegraphics[width=\textwidth]{SitiHajarAlZahara/5}
\caption{hasil nama dan data wordcount}
\label{gam:Hasil}
\end{figure}

\begin{figure}[!ht]
\includegraphics[width=\textwidth]{SitiHajarAlZahara/6}
\caption{hasil nama dan data wordcount}
\label{gam:Hasil}
\end{figure}
\end{enumerate}

\newday{\textbf{ 9 Desember 2022} program WordCount dengan java }
\begin{enumerate}
\item Kendala dan Solusi
\newline pada praktikum ini praktikan tidak terdapat kendala 
% jelaskan kendala dan penyebab yang dialami saat mengikuti praktikum serta solusi atau langkah-langkah yang telah dilakukan

\item Kesimpulan
\newline berhasil melakukan program WordCount dengan java.
% berikan kesimpulan dari praktikum yang telah dikerjkan
\begin{figure}[!ht]
\includegraphics[width=\textwidth]{SitiHajarAlZahara/7}
\caption{hasil wordcount java}
\label{gam:Hasil}
\end{figure}

\begin{figure}[!ht]
\includegraphics[width=\textwidth]{SitiHajarAlZahara/8}
\caption{hasil wordcount java}
\label{gam:Hasil}
\end{figure}

\begin{figure}[!ht]
\includegraphics[width=\textwidth]{SitiHajarAlZahara/9}
\caption{hasil wordcount java}
\label{gam:Hasil}
\end{figure}
\end{enumerate}

\newday{\textbf{ 15 Desember 2022} Instalasi Apacahe PySpark }
\begin{enumerate}
\item Kendala dan Solusi
\newline pada praktikum instalasi Apache PySpar ini praktikan tidak terdapat kendala 
% jelaskan kendala dan penyebab yang dialami saat mengikuti praktikum serta solusi atau langkah-langkah yang telah dilakukan

\item Kesimpulan
\newline berhasil melakukan instalasi PySpark dan tidak terdapat kendala apapun.
% berikan kesimpulan dari praktikum yang telah dikerjkan
\begin{figure}[!ht]
\includegraphics[width=\textwidth]{SitiHajarAlZahara/10}
\caption{hasil instalali Apache PySpark}
\label{gam:Hasil}
\end{figure}
\end{enumerate}

\include{laporan/SyarfaniAkbar}
\newthought{\textbf{Taravia Fauzah- 2020903430054 - TRKJ 3B}}

\newday{\textbf{1 - 2 Desember 2022} - Instalasi dan Konfigurasi Hadoop}
\begin{enumerate}
\item Kendala dan Solusi
% jelaskan kendala dan penyebab yang dialami saat mengikuti praktikum serta solusi atau langkah-langkah yang telah dilakukan
\newline praktikum pertama yaitu instalasi apache hadoop. selama mengerjakan praktikum mengikuti modul tidak ada kendala

\begin{figure}[!ht]
\includegraphics[width=\textwidth]{TaraviaFauzah/akses-localhost-9078}
\caption{hasil dari cek hadoop service}
\label{gam:perkuliahan15-9}
\end{figure}

\begin{figure}[!ht]
\includegraphics[width=\textwidth]{TaraviaFauzah/akses-localhost-8088}
\caption{hasil dari cek hadoop service}
\label{gam:perkuliahan15-9}
\end{figure}

\item Kesimpulan
% berikan kesimpulan dari praktikum yang telah dikerjkan
\newline Berhasil mendownload dan menginstal Apache hadoop dan sudah bisa di jalankan 
\end{enumerate}

\newday{\textbf{8 Desember 2022} - WordCount bawaan Hadoop}
\begin{enumerate}
\item Kendala dan Solusi
% jelaskan kendala dan penyebab yang dialami saat mengikuti praktikum serta solusi atau langkah-langkah yang telah dilakukan
\newline pada praktikum kali ini membuat program WordCount Bawaan Hadoop.pada saat praktikum tidak ada kendala hanya saja error karena salah menulis perintah.

\begin{figure}[!ht]
\includegraphics[width=\textwidth]{TaraviaFauzah/output}
\caption{hasil dari hadoop fs}
\label{gam:perkuliahan1-10}
\end{figure}

\begin{figure}[!ht]
\includegraphics[width=\textwidth]{TaraviaFauzah/output1}
\caption{hasil dari hadoop fs}
\label{gam:perkuliahan1-10}
\end{figure}

\begin{figure}[!ht]
\includegraphics[width=\textwidth]{TaraviaFauzah/output1_1}
\caption{hasil dari hadoop fs}
\label{gam:perkuliahan1-10}
\end{figure}

\item Kesimpulan
% berikan kesimpulan dari praktikum yang telah dikerjkan
\newline langkah praktikum ini adalah untuk memahami proses cara kerja pada hadoop dalam memproses data input sehingga menghasikan sebuah output. wordcount adalah program untuk menghitung jumlah kata dalam inpu.

\end{enumerate}

\newday{\textbf{9 Desember 2022} - WordCount dengan Java}
\begin{enumerate}
\item Kendala dan Solusi
% jelaskan kendala dan penyebab yang dialami saat mengikuti praktikum serta solusi atau langkah-langkah yang telah dilakukan
\newline pada praktikum kali ini membuat program WordCount dengan java.pada saat praktikum memiliki error tapi errornya di sebabkan tidak teliti saat menulis codingan yang ada di modul.solusinya harus lebih teliti saat mengerjakan codingan tersebut.

\begin{figure}[!ht]
\includegraphics[width=\textwidth]{TaraviaFauzah/hasil1}
\caption{hasil dari hadoop fs WordCount}
\label{gam:perkuliahan2-12}
\end{figure}

\begin{figure}[!ht]
\includegraphics[width=\textwidth]{TaraviaFauzah/hasil2}
\caption{hasil dari hadoop fs WordCount}
\label{gam:perkuliahan2-12}
\end{figure}

\begin{figure}[!ht]
\includegraphics[width=\textwidth]{TaraviaFauzah/hasil2_2}
\caption{hasil dari hadoop fs WordCount}
\label{gam:perkuliahan2-12}
\end{figure}

\item Kesimpulan
% berikan kesimpulan dari praktikum yang telah dikerjkan
\newline berhasil menjalan program WordCount dengan java

\end{enumerate}
\newthought{\textbf{Zulfahmi - 2020903430056 - TRKJ 3B}}

\newday{\textbf{1 - 2 Desember 2022} - Instalasi dan Konfigurasi Hadoop}

\begin{enumerate}

\item Kendala dan Solusi
\begin{enumerate}
    \item kendala
\begin{itemize}
    \item Salah saat memasang ssh dengan memasukan password sehingga tidak dapat melanjutkan jobdesk selanjutnya
\end{itemize}
    \item solusi
\begin{itemize}
    \item Memasang kembali ssh tanpa menggunakan password
\end{itemize}
\end{enumerate}

\item Kesimpulan
\newline
    Pada Instalasi Apache Hadoop membutuhkan ruang yang
    cukup besar untuk mengekstrak file Apache Hadoop dan ram
    minimal 2gb agar bekerja optimal.


\begin{figure}[!ht]
    \includegraphics[width=\textwidth]{Zulfahmi/1. install apache hadoop}
    \caption{Install apache hadoop}
    \label{gam:Hasil}
\end{figure}

\begin{figure}[!ht]
    \includegraphics[width=\textwidth]{Zulfahmi/2. konfigurasi hadoop 1}
    \caption{konfigurasi hadoop}
    \label{gam:Hasil}
\end{figure}

\begin{figure}[!ht]
    \includegraphics[width=\textwidth]{Zulfahmi/2. konfigurasi hadoop 2}
    \caption{konfigurasi hadoop}
    \label{gam:Hasil}
\end{figure}


\end{enumerate}

\clearpage
\newday{\textbf{8 Desember 2022} - WordCount bawaan Hadoop}
\begin{enumerate}
\item Kendala dan Solusi

\begin{itemize}
\item Tidak menemukan kendala apapun.
\end{itemize}

\item Kesimpulan
\newline
    Pada Hadoop terdapat program untuk menghitung jumlah kata 
    (WordCount) yang ada pada data. Sebagai praktikan melakukan input data terlebih
    dahulu, kemudian memprosesnya, sehingga menghasilkan data output.

\end{enumerate}

\begin{figure}[!ht]
    \includegraphics[width=\textwidth]{Zulfahmi/3. wordcount bawaan hadoop}
    \caption{hasil WordCount bawaan hadoop}
    \label{gam:Hasil}
\end{figure}

\clearpage
\newday{\textbf{9 Desember 2022} - WordCount dengan Java}
\begin{enumerate}
\item Kendala dan Solusi

\begin{itemize}
\item Tidak menemukan kendala apapun.
\end{itemize}

\item Kesimpulan
\newline
    Untuk hasil yang ditampikan sama dengan WordCount bawaan hadoop.

\end{enumerate}

\begin{figure}[!ht]
    \includegraphics[width=\textwidth]{Zulfahmi/4. wordcount dgn java}
    \caption{hasil WordCount Java}
    \label{gam:Hasil}
\end{figure}


\clearpage
\newday{\textbf{15 Desember 2022} - Instalasi Apache Spark}
\begin{enumerate}
\item Kendala dan Solusi

\begin{itemize}
\item Tidak menemukan masalah apapun
\end{itemize}


\item Kesimpulan
\newline
    Apache Spark adalah sebuah framework komputasi
    yang dapat digunakan untuk mengakses data, memproses
    data, menanyakan data serta menganalisis big data

\end{enumerate}

\begin{figure}[!ht]
\includegraphics[width=\textwidth]{Zulfahmi/5. dw apache spark}
\caption{hasil instalasi apache spark }
\label{gam:hasil instalasi spark}
\end{figure}

\clearpage
\newday{\textbf{16 Desember 2022} - WordCount Dengan Python}
\begin{enumerate}
\item Kendala dan Solusi

\begin{itemize}
    \item Tidak menemukan kendala apapun.
\end{itemize}
    

\item Kesimpulan
\newline Berhasil menjalankan program WordCountPython dengan baik, walaupun banyak kendala yang dialami.


\begin{figure}[!ht]
    \includegraphics[width=\textwidth]{Zulfahmi/6. wordcount dgn python}
    \caption{wordcount dgn python}
    \label{gam:hasil WordCountPython}
    \end{figure}

\end{enumerate}

\clearpage
\newday{\textbf{22 Desember 2022} - WordCount dengan PySpark}
\begin{enumerate}
\item Kendala dan Solusi

\begin{itemize}
    \item Tidak menemukan kendala apapun.
\end{itemize}

\item Kesimpulan
\newline Program Berhasil berjalan walau ada sedikit kendala yang saya alami.

\end{enumerate}

\begin{figure}[!ht]
    \includegraphics[width=\textwidth]{Zulfahmi/7. wordcount dgn pyspark}
    \caption{Hasil WordCount PySpark }
    \label{gam:hasil WordCountPyspark}
\end{figure}

\clearpage
\newday{\textbf{23 Desember 2022} - Machine Learning dengan PySpark}
\begin{enumerate}
\item Kendala dan Solusi

\begin{enumerate}
    \item kendala
\begin{itemize}
    \item Belum berhasil
\end{itemize}
    \item solusi
\begin{itemize}
    \item Terus mencoba
\end{itemize}
\end{enumerate}

\item Kesimpulan
\newline 

\end{enumerate}

\clearpage
\bibliographystyle{plain}
\bibliography{lab_notes}

\end{document}

%%%%%%%%%%%%%%%%%%%%%%%%%%%%%%%%%%%%%%%%%%%%%%%%%%%%%%%%
